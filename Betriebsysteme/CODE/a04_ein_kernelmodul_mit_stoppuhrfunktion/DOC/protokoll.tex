% PROTOKOLL BSP 02
\documentclass[
   draft=false
  ,paper=a4
  ,twoside=false
  ,fontsize=11pt
  ,headsepline
  ,BCOR10mm
  ,DIV11
  ,parskip=full+
]{scrartcl} % copied from Thesis Template from HAW

\usepackage[ngerman,english]{babel}
\usepackage[T1]{fontenc}
\usepackage[utf8]{inputenc}

\usepackage[german,refpage]{nomencl}

\usepackage{float}
\usepackage{enumitem}
\usepackage{hyperref} % for a better experience

\hypersetup{
   colorlinks=true % if false - links get colored frames
  ,linkcolor=black % color of tex intern links
  ,urlcolor=blue   % color of url links
}

\usepackage{amsmath}

\usepackage{array}   % for \newcolumntype macro
\newcolumntype{L}{>{$}l<{$}} % math-mode version of "l" column type
\newcolumntype{R}{>{$}r<{$}} % math-mode version of "r" column type
\newcolumntype{C}{>{$}c<{$}} % math-mode version of "c" column type

%\usepackage{listing}
\usepackage{caption}
%\usepackage{xcolor}
%\definecolor{bg}{rgb}{0.60,0.95,0.95}
%using minted because of the hashtag in bash
\usepackage{minted}
% c listing
\newminted{c}{fontsize=\small
             ,fontfamily=tt
             ,linenos
             ,frame=single
             } % \begin{ccode} ... \end{ccode}
\newmintedfile{c}{fontsize=\small
                 ,fontfamily=tt
                 ,linenos
                 ,frame=single
                 ,autogobble
                 } % \cfile{}
% Makefile listing
\newminted{make}{fontsize=\small
             ,fontfamily=tt
             ,linenos
             ,frame=single
             }  % \begin{makecode} ... \end{makecode}
\newmintedfile{make}{fontsize=\small
                    ,fontfamily=tt
                    ,linenos
                    ,frame=single
                    ,autogobble
                    } % \makefile{}
\sloppy
\clubpenalty=10000
\widowpenalty=10000
\displaywidowpenalty=10000

\begin{document}

\selectlanguage{ngerman}
\usemintedstyle{emacs}
% ----------------------------------------------------------------------------
% Titel (erst nach \begin{document}, damit babel bereits voll aktiv ist:
\titlehead{Betriebssysteme WS 2016 Praktikum 04}% optional
\subject{BS Praktikumsaufgabe 04}
\title{Ein Kernelmodul mit Stoppuhrfunktion}
\subtitle{Version 0.8}
\author{Alexander Mendel \\ Karl-Fabian Witte}
\date{erstellt am \today}% sinnvoll
%\publishers{Platz für Betreuer o.\,ä.}% optional
% ----------------------------------------------------------------------------
\maketitle% verwendet die zuvor gemachte Angaben zur Gestaltung eines Titels
\begin{abstract}
    Es soll ein Kernelmodul / Treiber  mit Stoppuhrfunktion auf der
    Virtuellen Maschine implementiert werden. Dabei zählt ein Gerät vorwärts
    \texttt{/dev/timerf} und eins rückwärts \texttt{/dev/timerr}. Als
    Zeiteinheit dienen die sogenannten \texttt{jiffes}. (Ein Linux eigener 
    Integerwert, welcher nach einer bestimmten Anzahl von Takten um Eins 
    erhöht wird.)
    Ziel ist es, ein wenig in den Abgründen des Linuxkernels zu schnuppern und
    sich mit den Konzepten der Treiber-/Modulprogrammierung vertraut 
    zu machen.
\end{abstract}
\tableofcontents
% ----------------------------------------------------------------------------
\flushleft

\section{Entwurf}
Es wird versucht sich an dem Buch "{}Linux Device Drivers"{} von Corbet,
Rubini und Kroah-Hartman zu halten. Zudem sollen sollen nur die zwei 
"{}Geräte"{} \texttt{timerr} und \texttt{timerf} erlaubt sein und diese auch 
nur jeweils in einmaliger Instanz. 

\subsection{Der Automat}
Die Geräte sind Automaten, die auf folgende Befehle reagieren
(z.B. \texttt{echo s > /dev/timerr}):
\begin{enumerate}
    \item[s :] Start:
        \begin{enumerate}
            \item[-] \texttt{timerf:} \texttt{READY}  $\to$ \texttt{RUNNING}
            \item[-] \texttt{timerr:} \texttt{LOADED} $\to$ \texttt{RUNNING}
        \end{enumerate}
    \item[p :] Pause: \texttt{RUNNING} $\to$ \texttt{PAUSE}
    \item[c :] Continue: \texttt{PAUSE} $\to$ \texttt{RUNNING}
    \item[r :] Reset: (alle internen Zeitwerte zurücksetzten):
        $\to$ \texttt{READY}
    \item[l$<$value$>$ :] Laden (nur \texttt{timerr}):
        \texttt{READY} $\to$ \texttt{LOADED}
\end{enumerate}
Dabei wird der Zustand der Geräte in der Funktion \texttt{timer\_write}
entsprechend geändert und die Werte werden gesetzt. Es wird mit
 \texttt{copy\_from\_user} in einen Kernelpuffer
gefüllt und dann mit \texttt{strchr}
ausgewertet.
Die Ausgabe des aktuellen Zeitwertes wird in \texttt{timer\_read} berechnet
und ausgegeben. Bei der Berechnung der Zeitwerte mit den jiffis ist zu
beachten, dass diese nicht negativ werden können, jedoch gibt es Macros im
jiffi header, welche den Vergleich zweier Zahlen zuverlässig im Zeitraum
einiger Tage abnehmen. Hier wird mit
\texttt{copy\_to\_user} die Ausgabe zum Benutzer geschickt. Sowohl
\texttt{timer\_write} als auch \texttt{timer\_read} haben einen lokalen
dynamischen Puffer, welcher mit einem Block auf einem Semaphoren (hier Mutex)
sicher allokiert und wieder entfernt wird.
Kontroverse Entwurfsentscheidung:
Wenn die Zeit der runterzählende Uhr abgelaufen ist, bleibt diese im Zustand 
RUNNING. 

\subsubsection{Hilfsfunktionen}
Sowohl für \texttt{timer\_read} als auch für \texttt{timer\_write} wurden
Hilfsfunktionen erstellt. Für \texttt{read} wurde die Berechnung der Zeit in
\texttt{calc\_time} gekapselt. Dort wird mit den eher starren Zeitvariablen
hantiert um die richtige Zeit zu berechnen, welche dann zurückgegeben wird.
Dabei wird auf das makro \texttt{time\_befor64} die Rückwärtsuhr verwendet,
um zu testen, ob die Zeit noch nicht erreicht wurde.

In der \texttt{timer\_write} wird mit \texttt{get event} zunächst der String
aus dem Userspace nach Befehlen durchsucht und ein entsprechendes
\texttt{enum event} zurückgegeben. Da der Load aufruf auch ein Argument
verlangt, wird eine Struktur mit \texttt{event} und \texttt{u64} zurückgegeben.
In der Funktion \texttt{update\_state} wird dann mit einer Fallunterscheidung
der Zustände der Automat entsprechend der Events umgeleitet und die
Zeitwerte gesetzt.
Nur beim Austritt aus dem Pausezustand wird der Offset erneut berechnet, von
der aus die Zeit in \texttt{read} berechnent wird. Ansonsten finden
alle Zeitberechnungen innerhalb \texttt{timer\_read} statt.

\section{Die Funktionen}
Damit unser Modul auch richtig funktioniert, müssen entsprechende Funktionen 
über eine bestimmte API in den Kernel eingebunden werden. Hier soll zunächst 
kurz über die Einbindung gesprochen werden und später werden die Abläufe in 
den Funktionen besprochen
Um das Module und de Treiber zu initialisieren, wird eine entsprechende 
Funktion einem dem Makro \texttt{module\_init} übergeben. Beim entfernen des 
Modules muss eine entsprechende Funktion dem Makro \texttt{module\_exit} 
übergeben werden.
\begin{ccode}
    module_init(timer_init);
    module_exit(timer_cleanup);
\end{ccode}
\texttt{timer\_init} wird beim laden des Modules ausgeführt, 
und \texttt{timer\_cleanup} entsprechend beim entfernen.

Wie sich der Treiber verhält, wird in der Struktur 
\texttt{struct file\_operations} festgelegt. Bei Zeichen orientierten Modulen 
wird \texttt{.owner} mit der symbolischen Konstanten \texttt{THIS\_MODULE} 
belegt. Unsere Devices sollen vom Benutzer gelesen werden \texttt{.read} 
(\texttt{cat /dev/timerf}) sowie auch Befehle entgegennehmen \texttt{.write} 
können (\texttt{echo <cmd> > /dev/timerr}). DA das Device als Datei in Linux 
existiert, muss es bei solchen operationen geöffnet \texttt{.open} und wieder
geschossen werden \texttt{.realaese}:
\begin{ccode}
struct file_operations timer_fops = {
    .owner   = THIS_MODULE,
    .read    = timer_read,    /* Berechnung und Ausgabe der Zeit an User*/
    .write   = timer_write,   /* Zustandsänderung */
    .open    = timer_open,    /* Uebergabe der Geraeteinfos ans file */
    .release = timer_release, /* eigentlich nix */
};
\end{ccode}

\subsection{init}
Die Funktion \texttt{timer\_init} soll das Modul und die Devices dem 
Betriebssystem nutzbar gemacht werden. Dafür wird zunächst die Majornummer 
dynamisch ermittelt, der Modulmutex erzeugt und darauf werden die Strukturen 
mit dem Informationen für die beiden Device mit der Funktion 
\texttt{timer\_dev\_init} gefüllt.

\subsubsection{timer\_dev\_init}
Die Struktur \texttt{timer\_dev} für die beiden Stoppuhren werden hier mit 
den Anfangwerten gefüllt. Zuerst wird der Speicher der Struktur alloziert.
Die beiden Treiber werden über ihre Minornummer unterschieden, welche hier 
auch übergeben wird. Ein Mutex soll zudem den dynamisch allozierten, lokalen
Kernelbuffer vor spontaner Terminierung schützen, damit der allozierte 
Speicherbereich wieder garantiert wieder freigegeben wird.
Das interessanteste ist die Character Device Struktur, welche hier erzeugt 
und der \texttt{timer\_dev} übergeben wird. Mit dieser Struktur wird das 
Device als Zeichenbezogenes Gerät von dem Betreibsystem behandelt.


\subsection{cleanup}
\texttt{timer\_cleanup} befreit den Speicher in umgekehrter Reihenfolge, wie 
er in \texttt{timer\_init} alloziert wurde.

\subsection{open}
\texttt{timer\_open} macht zunächst ein \texttt{down\_trylock} auf den 
Modulmutex \texttt{open\_sem} und gibt ggf. einen Fehlerwert zurück. Beim 
erfolgreichen "{}Locken"{} wird der Filestruktur die Datenstruktur 
\texttt{timer\_dev} des agierenden Gerätes bekannt gemacht. 

\subsection{release}
\texttt{timer\_release} befreit einfach den \texttt{open\_sem}.

\subsection{read}
\texttt{timer\_read} gibt die aktuelle Zeit der Stoppuhr wieder. 
Die Funktion wird erst nicht mehr aufgerufen, wenn eine negative Errornummer 
oder eine Null übergeben wird. Es wird jedoch nur der Zahlenwert der Rückgabe 
auch in den Userspace dann gedruckt. Deswegen wird die Funktion mindestens 
zweimal aufgerufen, wo jedoch beim letzten Mal Null zurückgegeben wird. 
Wir wissen, da die der Ausgabestring ein festest Format hat, wie viele Bytes 
übergeben werden müssen, und testen diese Anzahl mit dem Offset, welcher an 
die Stelle zeigt, an der wir das letzte mal aufgehört haben. 
Wenn der Offset kleiner als die erwünschte Anzahl ist, wird der Gerätemutex 
gelockt und ein lokaler Buffer alloziert. Dann wird die momentane Zeit in
\texttt{calc\_time} berechnet. Mit der Zeit wird in den Buffer die Message 
geschrieben und mit \texttt{copy\_to\_user} zum User transferiert. 
Danach wird der Offset neu berechnet, der Speicher des Buffers freigegeben und
der Mutex wieder gelöst.


\subsubsection{\texttt{calc\_time}}
In dieser Funktion wird die Zeit berechnet. Da wir meist statische 
Zustandsvariablen haben, wird hier wild gerechnet. Dafür werden vier lokale 
Variablen angelegt. Eine, welche die momentane Zeit beinhaltet \texttt{now}.
\texttt{time} ist die Variable, welche den Rückgabewert bunkert.\texttt{pause} 
ist ein Kontainer für die in Pausedauer, wenn sich das Gerät in der Pause 
befindet. Für die rückwärts laufende Uhr wurde noch eine Variable
\texttt{goal} definiert, welche den Zielzeitpunkt der Stoppuhr beinhaltet. 
Mit diesen Variablen wird dann die Zeit berechnet. Wenn das Ziel \texttt{goal}
erreicht wurde, soll nur Null zurückgegben werden. Dafür wird mit 
\texttt{time\_befor64} der Jiffiesbibliothek die momentane Zeit \texttt{now} 
gegen \texttt{goal} getetet.


\subsection{write}
Die Funktion \texttt{timer\_write} betritt erstmal einen vom "{}privaten"{} 
Gerätemutex geschützten Bereich und erzeugt einen Kernelbuffer, um die 
Informationen mit \texttt{copy\_from\_user} in diesen zu speicher.
Das erneute Aufrufen der Funktion hört erst bei einer negativen Errornummer 
oder bei der gesammten Anzahl der übermittleten Bytes auf. 
Bei erfolgreicher Übermittlung aller 
Bytes vom User- zum Kernelspace, wird die Funktion \texttt{update\_state} 
aufgerufen, welche zunächst aus dem Buffer mit der Funktion
\texttt{get\_event} den String in ein Event umwandelt. Danach wird der Status 
des Automaten neu gesetzt. Zurück in \texttt{timer\_write} wird der 
Speicherplatz des Buffers wieder frei gegeben und der Mutex hoch gesetzt. 

\subsubsection{\texttt{get\_event} und \texttt{update\_state}}
Für \texttt{get\_event} wird ein statisches und konstantes Array erzeugt, 
welches alle gültigen Befehle in der entsprechenden Reihenfolge des Enums 
\texttt{event} enthält. Über eine Schleife über das Array, wird mit Hilfe von
\texttt{strchr} in dem Buffer nach dem Befehl gesucht. Wird ein Befehlszeichen
identifiziert, wird ein Zähler inkrementiert und das Event (hier 
momentaner Laufvariablenwert) der Rückgabevariable zugewiesen. Da bei dem
Ladenevent ein Zahlenwert mit übergeben werden muss, wird dieser in so einem
Fall mit \texttt{sscanf} ausgelesen. Schlägt diese Funktion fehl, wird das 
Flag \texttt{is\_time\_valid} gelöscht.
Am ende wird nochmal geprüft, ob nur ein Befehl im String enthalten war 
und ob das Flag setzt ist. Wenn eines nicht zu traf, dann wird ein unbekanntes 
Event übergeben. Zu Debugzwecken wird hier der Bufferinhalt ausgegeben. 
Der Rückgabewert ist eine Struktur aus Eventwert und Zeitwert.
\texttt{updte\_state} überprüft für je nach Staus alle möglichen Events. Wenn
jedoch ein unbekanntes oder das ein reset Event passiert ist, werden diese 
vorher abgefangen und behandelt. Meist wird nur ein Zeitwert mit 
\texttt{get\_jiffi64} und der Status neu gesetzt. Nur bei der Transition vom 
Pausestatus zum Runningstatus wird die Dauer Pause auf die Offsetvariable 
dazugerechnt.


\section{Racing Conditions}
Wir haben mit dem Modulmutex und dem Devicemutex etwas übertrieben, da einer 
der beiden für unsere Zwecke gereicht hätte. Der Modulmutex ist hier der 
"{}Überflüssigere"{}, da wir keine Dateiglobalen Variablen haben, die von
beiden Geräten verändert werden. Doo

\section{Debugging}
Im Kernel zu debuggen ist relativ umständlich. Es kann nicht mit einem
Debugger gearbeitet werden. Es muss auf die gute alte "{}print everywhere and
everything"{} Methode zurückgegriffen werden. Dafür wurde ein Makro
\texttt{PDEBUG} erstellt, welches mit der Kernel eigenen Format-Print-Funktion
\texttt{printk} und dem dazugehörigen Makro \texttt{KERN\_DEBUG} Meldungen an
die Datei \texttt{/var/log/message} übergibt, welche wir dann in einer zusätzlichen
Konsole mit \texttt{tail -f} "{}live"{} beobachten. Es werden mithilfe von
\texttt{grep} oder \texttt{awk} die wichtigen Information heraus gefiltert.
Es wurde zu dem eine
Funktion \texttt{dev\_dump} erstellt, um den momentanen Status der Timer zu
erfahren. Diese wird immer nach jeder Statusänderung des Automaten ausgegeben.
Zudem wir in der Funktion \texttt{get\_event} der Buffer ausgegbeen, damit man 
diese Evententscheidung nachvollziehen kann. In \texttt{calc\_time} werden die
lokalen Variablen ausgegeben, da die Berechnung nachvollziehbar bleibt. 
Zudem werden bei jedem Funktionsaufruf der Name der Funktion ausgerufen, um 
den momentanen Programmzeiger/Verlauf zu verfolgen.
Noch besser wäre, beschreibende Nachrichten in den Debugmitteilungen zu 
schreiben, die auch ein paar mehr Informationen des Zustandes der lokalen 
Variablen beinhalten.


\section{Installierung}
Das \texttt{Makefile} (welches unbedingt mit großen "{}M"{}, sonst findet er
das nicht vom Kernel aus) erstellt mit dem Befehl \texttt{make} das Modul. Die
Compilerfalgs werden mit der vom Kernel eigenen Variable \texttt{ccflags-y}
übergeben.
Für das mühselige laden und löschen der "{}Geräte"{} wurde ein
Shellskript \texttt{timer.sh} erstellt, welches mit den Optionen
\texttt{--load, --unload} diese Arbeit abnimmt. Dabei ruft \texttt{load} die
\texttt{unload}-Methode indirekt auf, um alte Geräte und Treiber zu entfernen.
Danach wird dann der Treiber \texttt{insmod timer.ko} in den Kernel eingefügt,
mit Hilfe von Linuxterminalhacks wird die Majornummer herausgeschrieben und
mit dieser via \texttt{mknod} und der Minornummer die Geräte installiert.
Und da das noch nicht bequem genug ist, kann man als Superuser mit
\texttt{make load} bzw. \texttt{make unload} das laden auch so bewerkstelligen.
%\section{Quellcode}


\section{Kommentare}
Für die genauere und detailliertere Betrachtung, haben wir den Code extra 
viele Kommentare beigefügt, welche bei der Abnahme fehlten.


\section{Sonstiges}

Es wird für die Geräte die folgende Struktur verwendet, welche hier kurz
aufgeführt wird.
\begin{ccode}
struct time {
    u64 offset;   /* startzeit + pausenzeiten */
    u64 enter_pause; /* zeitpunkt beim Eintritt in den Pausenzustand */
    u64 load;       /* geladene Zeit */
}

struct timer_dev {
    int minor;              /* timerf or timerr */
    char name[TIMER_NAME_LEN]  /* string for name */
    struct time time;       /* time variables */
    enum state state; /* current state */

    struct semaphore sem;   /* mutex to avoid race conditions */
    struct cdev cdev;       /* stuff for char devs */
};

\end{ccode}
mit dem Befehl \texttt{./timer.sh --setup} kann man sich zwei Konsolen öffnen,
wovon eine das Logfile anzeigt und das andere für die Befehlseingabe bereit
steht.  \texttt{./timer.sh --test} wird ein testlauf durchgeführt, welcher 
nach unseren Ermessen die korrekte Funktionalität präsentiert, da fast alle 
(nicht immer die nicht erlaubten/bekannten) Befehle in jedem Zustand 
durchprobiert wuden (Robusthaitstest).

Es wird versucht, gänzlich auf \texttt{goto} zu verzichten, obwohl dieses im
Buch für das Zusammenspiel zwischen \texttt{init} und \texttt{cleanup} sehr
gut umgesetzt wurde. Für uns hat sich das \texttt{goto} noch nicht als großer
Nutzen gezeigt, da wir ja auch nicht auf Performance aus sind.


\end{document}
% vim: set spell spelllang=de :EOF
