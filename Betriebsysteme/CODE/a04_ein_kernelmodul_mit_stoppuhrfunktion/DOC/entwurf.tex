% PROTOKOLL BSP 02
\documentclass[
   draft=false
  ,paper=a4
  ,twoside=false
  ,fontsize=11pt
  ,headsepline
  ,BCOR10mm
  ,DIV11
  ,parskip=full+
]{scrartcl} % copied from Thesis Template from HAW

\usepackage[ngerman,english]{babel}
\usepackage[T1]{fontenc}
\usepackage[utf8]{inputenc}

\usepackage[german,refpage]{nomencl}

\usepackage{float}
\usepackage{enumitem}
\usepackage{hyperref} % for a better experience

\hypersetup{
   colorlinks=true % if false - links get colored frames
  ,linkcolor=black % color of tex intern links
  ,urlcolor=blue   % color of url links
}

\usepackage{amsmath}

\usepackage{array}   % for \newcolumntype macro
\newcolumntype{L}{>{$}l<{$}} % math-mode version of "l" column type
\newcolumntype{R}{>{$}r<{$}} % math-mode version of "r" column type
\newcolumntype{C}{>{$}c<{$}} % math-mode version of "c" column type

%\usepackage{listing}
\usepackage{caption}
%\usepackage{xcolor}
%\definecolor{bg}{rgb}{0.60,0.95,0.95}
%using minted because of the hashtag in bash
\usepackage{minted}
% c listing
\newminted{c}{fontsize=\small
             ,fontfamily=tt
             ,linenos
             ,frame=single
             } % \begin{ccode} ... \end{ccode}
\newmintedfile{c}{fontsize=\small
                 ,fontfamily=tt
                 ,linenos
                 ,frame=single
                 ,autogobble
                 } % \cfile{}
% Makefile listing
\newminted{make}{fontsize=\small
             ,fontfamily=tt
             ,linenos
             ,frame=single
             }  % \begin{makecode} ... \end{makecode}
\newmintedfile{make}{fontsize=\small
                    ,fontfamily=tt
                    ,linenos
                    ,frame=single
                    ,autogobble
                    } % \makefile{}
\sloppy
\clubpenalty=10000
\widowpenalty=10000
\displaywidowpenalty=10000

\begin{document}

\selectlanguage{ngerman}
\usemintedstyle{emacs}
% ----------------------------------------------------------------------------
% Titel (erst nach \begin{document}, damit babel bereits voll aktiv ist:
\titlehead{Betriebssysteme WS 2016 Praktikum 04}% optional
\subject{BS Praktikumsaufgabe 04}
\title{Ein Kernelmodul mit Stoppuhrfunktion}
\subtitle{Version 0.6 - Abgabe am \today 16:00}
\author{Alexander Mendel \\ Karl-Fabian Witte}
\date{erstellt am \today}% sinnvoll
%\publishers{Platz für Betreuer o.\,ä.}% optional
% ----------------------------------------------------------------------------
\maketitle% verwendet die zuvor gemachte Angaben zur Gestaltung eines Titels
\begin{abstract}
    Es soll ein Kernelmodul / Treiber  mit Stoppuhrfunktion auf der
    Virtuellen Maschine implementiert werden. Dabei zählt ein Gerät vorwärts
    \texttt{/dev/timerf} und eins rückwärts \texttt{/dev/timerr}. Als
    Zeiteinheit dienen die sogenannten \texttt{jiffes} (Takt der Maschine).
    Ziel ist es, ein wenig in den Abgründen des Linuxkernels zu schnuppern und
    sich mit den Konzepten der Treiberprogrammierung vertraut zu machen.
\end{abstract}
\tableofcontents
% ----------------------------------------------------------------------------
\flushleft
\section{Entwurf}
Es wird versucht sich an dem Buch "{}Linux Device Drivers"{} von Corbet,
Rubini und Kroah-Hartman zu halten. Es sollen nur die zwei Geräte
\texttt{timerr} und \texttt{timerf} erlaubt sein und diese auch nur jeweils in
einmaliger Instanz.

\subsection{init und cleanup}
Jedes Modul wird über eine bestimmt Funktion initialisiert, die dem Makro
\texttt{module\_init(fkt)} zum richtigen Einpflanzen in die Kernelerde
übergeben wird. Zum Ausgraben wird dem Makro eine andere Funktion
\texttt{module\_exit(fkt)} übergeben.

\subsubsection{timer\_init}
    Dies ist unsere Initialisationsfunktion. Zunächst wird eine Majornummer
    allokiert und die statischen Variablen übergeben. Danach werden die
    Geräte installiert, indem die Hilfsmethode \texttt{timer\_dev\_init}
    für jeweils ein Device-Node ausgerufen wird. Dort wird der Speicher
    der \texttt{timer\_dev} bestimmt und die Initialisierungswerte werden gesetzt.
    Dazu wird die \texttt{cdev}-Struktur initialisiert, damit unser Gerät
    auch mit der Umgebung agieren kann.
\subsubsection{timer\_cleanup}
    Hier werden in umgekehrter Reihenfolge die erzeugten Objekte wieder
    zerstört.

\subsection{Der Automat}
Die Geräte sind Automaten, die auf folgende Befehle reagieren
(z.B. \texttt{echo <cmd> > /dev/timerr}):
\begin{enumerate}
    \item[s :] Start:
        \begin{enumerate}
            \item[-] \texttt{timerf:} \texttt{READY}  $\to$ \texttt{RUNNING}
            \item[-] \texttt{timerr:} \texttt{LOADED} $\to$ \texttt{RUNNING}
        \end{enumerate}
    \item[p :] Pause: \texttt{RUNNING} $\to$ \texttt{PAUSE}
    \item[c :] Continue: \texttt{PAUSE} $\to$ \texttt{RUNNING}
    \item[r :] Reset: (alle internen Zeitwerte zurücksetzten):
        $\to$ \texttt{READY}
    \item[l$<$value$>$ :] Laden (nur \texttt{timerr}):
        \texttt{READY} $\to$ \texttt{LOADED}
\end{enumerate}
Dabei wird der Zustand der Geräte in der Funktion \texttt{timer\_write}
entsprechend geändert und die Werte werden gesetzt. Es wird mit
 \texttt{copy\_from\_user} in einen Kernelpuffer
gefüllt und dann mit \texttt{strchr}
ausgewertet.
Die Ausgabe des aktuellen Zeitwertes wird in \texttt{timer\_read} berechnet
und ausgegeben. Bei der Berechnung der Zeitwerte mit den jiffis ist zu
beachten, dass diese nicht negativ werden können, jedoch gibt es Macros im
jiffi header, welche den Vergleich zweier Zahlen zuverlässig im Zeitraum
einiger Tage abnehmen. Hier wird mit
\texttt{copy\_to\_user} die Ausgabe zum Benutzer geschickt. Sowohl
\texttt{timer\_write} als auch \texttt{timer\_read} haben einen lokalen
dynamischen Puffer, welcher mit einem Block auf einem Semaphoren (hier Mutex)
sicher allokiert und wieder entfernt wird.

\subsubsection{Hilfsfunktionen}
Sowohl für \texttt{timer\_read} als auch für \texttt{timer\_write} wurden
Hilfsfunktionen erstellt. Für \texttt{read} wurde die Berechnung der Zeit in
\texttt{calc\_time} gekapselt. Dort wird mit den eher starren Zeitvariablen
hantiert um die richtige Zeit zu berechnen, welche dann zurückgegeben wird.
Dabei wird auf das makro \texttt{time\_befor64} die Rückwärtsuhr verwendet,
um zu testen, ob die Zeit noch nicht erreicht wurde.

In der \texttt{timer\_write} wird mit \texttt{get event} zunächst der String
aus dem Userspace nach Befehlen durchsucht und ein entsprechendes
\texttt{enum event} zurückgegeben. Da der Load aufruf auch ein Argument
verlangt, wird eine Struktur mit \texttt{event} und \texttt{u64} zurückgegeben.
In der Funktion \texttt{update\_state} wird dann mit einer Fallunterscheidung
der Zustände der Automat entsprechend der Events umgeleitet und die
Zeitwerte gesetzt.
Nur beim Austritt aus dem Pausezustand wird der Offset erneut berechnet, von
der aus die Zeit in \texttt{read} berechnent wird. Ansonsten finden
alle Zeitberechnungen innerhalb \texttt{timer\_read} statt.


\subsection{file\_operations}
Hier soll einmal kurz \texttt{file\_operations} gezeigt werden, welche an
die \texttt{cdev} weitergegeben wird, damit das Betriebssystem auch weiß,
was zu tun ist.
\begin{ccode}
struct file_operations timer_fops = {
    .owner   = THIS_MODULE,
    .read    = timer_read,   /* Berechnung und Ausgabe der Zeit an User*/
    .write   = timer_write,  /* Zustandsänderung */
    .open    = timer_open,   /* Uebergabe der Geräteinfos ans file */
    .release = timer_release, /* eigendlich nix */
};
\end{ccode}


\subsection{Debugging}
Im Kernel zu debuggen ist relativ umständlich. Es kann nicht mit einem
Debugger gearbeitet werden. Es muss auf die gute alte "{}print everywhere and
everything"{} Methode zurückgegriffen werden. Dafür wurde ein Makro
\texttt{PDEBUG} erstellt, welches mit der Kernel eigenen Format-Print-Funktion
\texttt{printk} und dem dazugehörigen Makro \texttt{KERN\_DEBUG} Meldungen an
die Datei \texttt{/var/log/message} übergibt, welche wir dann in einer zusätzlichen
Konsole mit \texttt{tail -f} "{}live"{} beobachten. Es werden mithilfe von
\texttt{grep} oder \texttt{awk} die wichtigen Information heraus gefiltert.
Es wurde zu dem eine
Funktion \texttt{dev\_dump} erstellt, um den momentanen Status der Timer zu
erfahren.

\subsection{Installierung}
Das \texttt{Makefile} (welches unbedingt mit großen "{}M"{}, sonst findet er
das nicht vom Kernel aus) erstellt mit dem Befehl \texttt{make} das Modul. Die
Compilerfalgs werden mit der vom Kernel eigenen Variable \texttt{ccflags-y}
übergeben.
Für das mühselige laden und löschen der "{}Geräte"{} wurde ein
Shellskript \texttt{timer.sh} erstellt, welches mit den Optionen
\texttt{--load, --unload} diese Arbeit abnimmt. Dabei ruft \texttt{load} die
\texttt{unload}-Methode indirekt auf, um alte Geräte und Treiber zu entfernen.
Danach wird dann der Treiber \texttt{insmod timer.ko} in den Kernel eingefügt,
mit Hilfe von Linuxterminalhacks wird die Majornummer herausgeschrieben und
mit dieser via \texttt{mknod} und der Minornummer die Geräte installiert.
Und da das noch nicht bequem genug ist, kann man als Superuser mit
\texttt{make load} bzw. \texttt{make unload} das laden auch so bewerkstelligen.
%\section{Quellcode}

\section{Sonstiges}

Es wird für die Geräte die folgende Struktur verwendet, welche hier kurz
aufgeführt wird.
\begin{ccode}
struct time {
    u64 offset;
    u64 enter_pause;
    u64 load;
}

struct timer_dev {
    int minor;              /* timerf or timerr */
    char name[TIMER_NAME_LEN]  /* string for name */
    struct time time;       /* time variables */
    enum state state; /* current state */

    struct semaphore sem;   /* mutex to avoid race conditions */
    struct cdev cdev;       /* stuff for char devs */
};

\end{ccode}
mit dem Befehl \texttt{/timer.sh --setup} kann man sich zwei Konsolen öffnen,
wovon eine das Logfile anzeigt und das andere für die Befehlseingabe bereit
steht.

Es wird versucht, gänzlich auf \texttt{goto} zu verzichten, obwohl dieses im
Buch für das Zusammenspiel zwischen \texttt{init} und \texttt{cleanup} sehr
gut umgesetzt wurde. Für uns hat sich das \texttt{goto} noch nicht als großer
Nutzen gezeigt, da wir ja auch nicht auf Performance aus sind.


\end{document}
% vim: set spell spelllang=de :EOF
