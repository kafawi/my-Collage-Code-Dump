% PROTOKOLL BSP 02
\documentclass[
   draft=false
  ,paper=a4
  ,twoside=false
  ,fontsize=11pt
  ,headsepline
  ,BCOR10mm
  ,DIV11
  ,parskip=full+
]{scrartcl} % copied from Thesis Template from HAW

\usepackage[ngerman,english]{babel}
\usepackage[T1]{fontenc}
\usepackage[utf8]{inputenc}

\usepackage[german,refpage]{nomencl}

\usepackage{float}
\usepackage{enumitem}
\usepackage{hyperref} % for a better experience

\hypersetup{
   colorlinks=true % if false - links get colored frames
  ,linkcolor=black % color of tex intern links
  ,urlcolor=blue   % color of url links
}

\usepackage{array}   % for \newcolumntype macro
\newcolumntype{L}{>{$}l<{$}} % math-mode version of "l" column type
\newcolumntype{R}{>{$}r<{$}} % math-mode version of "r" column type
\newcolumntype{C}{>{$}c<{$}} % math-mode version of "c" column type

%\usepackage{listing}
\usepackage{caption}
%\usepackage{xcolor}
%\definecolor{bg}{rgb}{0.60,0.95,0.95}
%using minted because of the hashtag in bash
\usepackage{minted}
% c listing
\newminted{c}{fontsize=\small
             ,fontfamily=tt
             ,linenos
             ,frame=single
             } % \begin{ccode} ... \end{ccode}
\newmintedfile{c}{fontsize=\small
                 ,fontfamily=tt
                 ,linenos
                 ,frame=single
                 ,autogobble
                 } % \cfile{}
% Makefile listing
\newminted{make}{fontsize=\small
             ,fontfamily=tt
             ,linenos
             ,frame=single
             }  % \begin{makecode} ... \end{makecode}
\newmintedfile{make}{fontsize=\small
                    ,fontfamily=tt
                    ,linenos
                    ,frame=single
                    ,autogobble
                    } % \makefile{}
\sloppy
\clubpenalty=10000
\widowpenalty=10000
\displaywidowpenalty=10000

\begin{document}

\selectlanguage{ngerman}
\usemintedstyle{emacs}
% ----------------------------------------------------------------------------
% Titel (erst nach \begin{document}, damit babel bereits voll aktiv ist:
\titlehead{Betriebsysteme WS 2016 Praktikum 02}% optional
\subject{BS Praktikumsaufgabe 02}
\title{Thread Koordination: Die trainierenden PhilosophInnen}
\subtitle{Version 0.2 - Abgabe am 7. November 2016 16:00}
\author{Alexander Mendel \\ Karl-Fabian Witte}
\date{erstellt am \today}% sinnvoll
%\publishers{Platz für Betreuer o.\,ä.}% optional
% ----------------------------------------------------------------------------
\maketitle% verwendet die zuvor gemachte Angaben zur Gestaltung eines Titels
\begin{abstract}
  Zu Anfang wird ein Makefile zu einem gegebenen C-Datei
  erstellt und somit das Grundprinzip von (GNU)Make gelernt.
  Am Ende soll die automatische Generierung der Abhängigkeiten der C-Datein
  von den Headerdatein für das Makefile erklärt werden. In der Hauptaufgabe
  wird die Synchronisation zwischen Threads geübt.
  Dabei wird unter anderem das Prinzip eines Monitors verwendet.
\end{abstract}
\tableofcontents
% ----------------------------------------------------------------------------
\flushleft
\section{Vorübung: Makefile}
Um später einmal alle Quelldateien für die Kompilierung und Verlinkung zu
organisieren, wird hier (GNU)Make geübt. Im Prinzip ist es eine kleine
Skriptsprache, welche die Kompilierung koordiniert und somit mit einfachen
Befehlen alle abhängigen Quelldateien überprüft und gegebenenfalls diesen
Zweig wieder übersetzt.
Zum üben wurde uns nur eine Quelldatei gegeben und wir haben das Makefile,
wie hier dargestellt, realisiert:
\begin{makecode}
  EXE = critsect02

  CC = /usr/bin/gcc

  CFLAG  = -pthread
  CFLAG += -g
  CFLAG += -Wall

  LDFLAG = -lpthread

  SRC = critsect02.c

  # targets (all is default, because it is on top.)
  all: $(EXE)

  run: $(EXE)
  	./$(EXE)

  $(EXE): $(SRC)
  	$(CC) $(CFLAG) $(LDFLAG) -o $@ $<

  clean:
  	rm -f $(EXE)
\end{makecode}

\subsection{Automatische Generierung der Headerabhängigkeiten}
Der C-Kompilierer von GNU kann uns sehr viel Arbeit abnehmen: Er erspart uns
mit einem kleinen Trick, dass wir jeden Header einzeln zu der jeweiligen
Objektdatei zuordnen müssen. So können wir nämlich auch die Wildcardmethode
\texttt{\%.o:\%.c} weiter verwenden und trotzdem die Headerabhängigkeiten
berücksichtigen.

Mit dem Flag \texttt{-M} werden alle Headerdatein der jeweiligen Objektdatei
zugeordnet und diese im Makefileformat
(\texttt{traget.o: header1.h header2.h}) auf die Ausgabe geleitet.
Diese Ausgabe leitet man in eine Datei, die in das Makefile eingefügt wird.
Die beschrieben Stellen unseres Makefiles sind hier einmal dargestellt:
\begin{makecode}
  DEPENDFILE = .depend

  ifeq ($(DEPENDFILE),$(wildcard $(DEPENDFILE)))
  -include $(DEPENDFILE)
  endif

  dep depend $(DEPENDFILE):
  	$(CC) $(CFLAGS) -MM *.c > $(DEPENDFILE)

\end{makecode}
\subsubsection*{-MM}
Dieses Flag erstellt nur die Abhängigkeiten der privaten Header. Die
Bibliotheksheader werden nicht mit aufgeführt.

\subsubsection*{\$(wildcard)}
Diese Operation erstellt eine Liste der Dateien im selben Ordner wie das
Makefile, wenn als Argument ein String mit Wildcardzeichen übergeben wird.
Wenn man genau hin guckt, überprüft diese Bedingung in unserem Quellcode nur,
ob sich eine Abhängigkeitsdatei im Ordner befindet und bindet diese ggf. ein.

\subsubsection*{Fehler bei der erneuten Berechnung der Abhängigkeiten}
Dieser Fehler tritt auf, wenn man keine Rechte für die neu erstellte
Abhängigkeitsdatei hat, was auf der virtuellen Maschine im eingehängten
Teilordner der Fall zu sein scheint.

\subsubsection*{Weitere Probleme}
Es scheint einen Haufen von Problemen  zu geben, wenn es um diese
\texttt{Dependance} geht. Deswegen wird das Kapitel nur vorläufig
geschlossen und weiter verfolgt.

% -----------------------------------------------------------------------------
\section{Thread-Koordination in der Muckibude}
Der folgende Abschnitt wurde kaum verändert aus unserem Entwurf entnommen.

In dieser Aufgabe soll die Thread-Koordination mit den Bibliotheken der
POSIX Threads und Semaphoren unter Linux geübt werden. Dafür schicken wir 5
Philosophen (Threads) in die Muckibude. Wie wir die Koordination der Gewichte
(Daten) zwischen den Philosophen realisieren, soll in den folgenden Abschnitten
erläutert werden.

\subsection{Überblick}
Dieser Abschnitt dient dazu, um sich schnell ein Bild machen zu können.

Das Programm wird aus folgenden Quellcodedatein zusammengestellt:
\begin{enumerate}
  \item[parameter.h] Hier sind alle Konstanten für die Implementierung
    definiert.
  \item[gym.c] Realisierung eines Monitors mit zwei Monitorfunktionen und den
    (empfindlichen) Daten: den Gewichten. Zudem weitere Hilfsfunktionen.
  \item[gym.h] Das Interface für Funktionen und Typedefinitionen.
  \item[philo.c] Die Threadfunktion, welche die Philosophen zu einem Zyklus
    des Gewichte Holen, Heben, Weglegen und Ausruhen treibt. Zudem Deklaration
    des globalen Befehlfeldes.
  \item[philo.h] Das Interface und die externe Bekanntmachung des globalen
   Befehlfeldes.
  \item[global.c] Die Funktion \texttt{handle\_error} wird hier implementiert.
    Damit wird der Rückgabewert von einer Bibliotheksfunktion ausgewertet,
    bei einem Fehler die entsprechende Meldung ausgegeben und das Programm
    beendet.
  \item[global.h] Das Interface für \texttt{handle\_error}. Zudem werden hier
    \texttt{TRUE} und \texttt{FALSE} definiert.
  \item[main.c] Die Mainfunktion mit weiteren Hilfsfunktionen. Der Mainthread
    koordiniert die Initialisierung und Zerstörung der anderen Threads und
    deren Objekten. Zudem steuert der Mainthread die interaktive Annahme der Befehle
    über die Konsole.
\end{enumerate}

Es sollen kurz die Prototypen der wichtigsten Funktionen aufgelistet werden:
\begin{ccode}
//     GYM
void init_gym_obj();  // init the stock, bags, pthread_mutex and pthread_cond

void free_gym_obj();  // destroys the pthread_mutex and pthread_cond
void free_gym_threads(); //  set is_quit = TRUE and pthread_cond_broadcast

int get_weights( int tid );  // monitor func
int put_weights( int tid );  // monitor func
int display_status( void );  // is called in monitor functions

int fill_bag( weight_bag_t * bag);  // get the weights form stock into bag
void flush_bag(weight_bag_t * bag); // put the weights back in stock

//     PHILO
void init_philo_obj(void);  // init philostructur, semaphore
void free_philo_obj(void);  // destroy the semaphore

void * philothread (void * ptr_tid); // thread function, no return, param tid

void set_train_status(train_state_t state, int tid); // used by monitor func

void unblock_philo(int tid);  // used to unblock the thread

//     GLOBAL
void handle_error(  int result,         // an universal tool
                    int errnum,
                    char const *mssge);
\end{ccode}


Und hier der Parameterheader:
\begin{ccode}
  #define NPHILO      5
  //#define PHILO_NAMES { "Anne", "Bernd", "Clara", "Dirk", "Emma" }
  #define PHILO_KG    {      6,       8,      12,     12,     14 }

  #define NWEIGHTS_T 3
  #define WEIGHT_KG  { 2, 3, 5 }
  #define WEIGHT_QTY { 4, 4, 5 }

  #define REST_LOOP   1000000000
  #define WORKOUT_LOOP 500000000
\end{ccode}


\subsubsection{Programmablauf}
\textbf{Der Mainthread} erzeugt bzw. initialisiert alle Objekte. Dann werden
die Pthreads (Philosophen) gestartet. Danach ist der Mainthread in der
Controllerschleife. In dieser Endlosschleife wartet der
Mainthread auf eine Eingabe aus der \texttt{stdin} und überprüft diesen
auf die Befehle:
\begin{enumerate}
  \item[q oder Q] Beendet das Programm sauber.
  \item[\texttt{tid}b] Blockier Thread Nummer tid.
  \item[\texttt{tid}u] Befreit Thread Nummer tid.
  \item[\texttt{tid}p] Thread Nummer tid überspringt die REST und WORK\_OUT
   Schleifen.
\end{enumerate}

Ein akzeptierte Befehl wird entschlüsselt und
bearbeitet. Wenn der Befehl zum Schließen des Programms erkannt wurde, werden
alle Philosophethreads gebeten, sich zu beenden (über das globale
Befehlsfeld). Alle blockierten Threads werden befreit, indem man die
Blockierungsobjekte freigibt. Der Mainthread wartet nun darauf, dass alle
Threads terminierten und zerstört alle erstellten Objekte. Erst dann beendet
sich der Mainthread und terminiert somit das Programm.

\textbf{Einem Philothread} wird die Nummer (\texttt{tid}) mitgegeben, mit
welcher er auf die für ihn wichtigen Felder zugreifen und diese manipulieren kann.
Er durchläuft in einer Endlosschleife, in der er vier Funktionen pro Schleife
ausführt. Zwei davon sind fast leere Zählerschleifen, in den er auf die
Befehle in dem globalen Array reagiert. Die anderen beiden Funktionen werden
durch das Monitormodell realisiert, um die Konsistenz der Daten (Gewichte)
wahren zu können. In diesen beiden Funktionen ignoriert der Philothread zunächst die
Befehle des Mainthreads. In den Funktionen des Monitors wird beim Eintreten in
und beim Austreten aus dem geschützten Bereich innerhalb diesem der Status auf
der Konsole ausgegeben.

\subsection{Sichtbarkeit der Objekte}
  Da es sich um eine statische Anzahl an Threads und Objekten handelt, können
  die Objekte ebenfalls statisch sein. Unsere Synchronisationsobjekte haben
  zudem Dateisichtbarkeit. Somit wird für jede C-Datei eine Initialisierungs-
  und eine Zerstörungsfunktion implementiert, welche dann von dem Mainthread
  aufgerufen wird.
  Das globale (zwischen mehreren C-Datein sichtbar) Befehlsfeld ist von
  main.c und von philo.c aus sichtbar.

  Wir haben uns gegen dynamischen HEAP-Speicher entschieden, da dies keinen
  Sinn ergebe, da kein Mehrwert für uns zu erkennen war.

\subsubsection{Struckturen der Objekte}

Gerade die zu schützenden Objekt sind hier von großer Bedeutung: die Gewichte.
Dabei verwaltet gym.c alle Gewichtobjekte (weight), auch die der Philosophen.
Beim ersten Blick ist das zunächst komisch, warum der Philosoph selber nicht
weiß, wie viel er für sein Training an kg bekommt. Die Muckibude brauch als
einziges Programmteil diesen Wert, um die Gewichte aus dem Lager 
(\texttt{stock}) in die für die Philosophen bereitgestellten Gewichtetasche  
(\texttt{bag}) zu füllen:

Deswegen hat die Muckibude folgende Struckturen im statischen privaten
Sichtbarbeitsfeld:
\begin{ccode}
  typedef struct {
      int kg;        // type
      int qty;       // quantity
  } weight_t;

  struct{
      weight_t weights[ NWEIGHTS_T ]; // slots
      int total_kg;                   // total amount of all weights in the gym
      int available_kg;               // amount what is left
  } stock;

  typedef struct {
      weight_t weights [ NWEIGHTS_T ];   // slots
      int required_kg;                   // required for a effective trainigs
  } weight_bag_t;

  weight_bag_t bags[NPHILO];             // for every philo one bag

  pthread_mutex_t gym_mtx;
  pthread_cond_t gym_cv;
\end{ccode}

Die Philosphen haben primär die Information, in welchem Status sie sich
befinden:
\begin{ccode}
typedef enum {
    QUIT=20,
    BLOCKED,
    NORMAL
} cmd_state_t;

typedef enum {
    GET_WEIGHTS=10,
    WORKOUT,
    PUT_WEIGHTS,
    REST
} train_state_t;

typedef struct {
    sem_t block_sem;
    cmd_state_t cmd_state;
    train_state_t train_state;
    int tid;
} philo_t;

extend char cmd_arr[];
\end{ccode}

\subsection{Die Muckibude als Monitor}
  Die Muckibude \texttt{gym.c} hat als Synchronisationsobjekte einen POSIX
  Mutex \texttt{gym\_mtx} und eine POSIX Bedingungsvariable \texttt{gym\_cv}.
  Beim Eintreten und Verlassen des geschützten Bereich, wird jeweils die
  Der Trainierstatus des Philothreats entsprechend geändert und
  \texttt{display\_status} aufgerufen. \texttt{display\_status} überprüft,
  ob auch kein Gewicht verloren gegangen ist und gibt ggf. eine Fehlermeldung
  aus.

  Nur die Threads, die auf die sich Gewichte holen wollen
  (\texttt{GET\_WEIGHTS}), können auf Grund der beschränkten verfügbaren Gewichte die
  erforderlichen Gewichte nicht holen. Somit warten
  sie auf der POSIX Bedingungsvariable in einer Whileschleife und geben den
  geschützen Bereich wieder frei. Wenn aber einige Gewicht von einem anderen
  Thread wieder zurückgelegt wurden (\texttt{PUT\_WEIGHTS}), wird vor dem
  Auftreten aus dem sicheren Bereich alle auf Bedingungsvariablen sitzenden
  Threads geweckt, welche dann wiederum prüfen können, ob sie diesmal die
  Gewichte kriegen können.

  Das folgende Listing zeigt die
  beiden Monitorfunktionen. Die zweite Bedingung \texttt{is\_quit} ist für 
  die Freisetzung der blockierenden Threads, nachdem der Beendenbefehl erteilt wurde.

  \begin{ccode}
/* -------------------------------------------------------------GET_WEIGHTS */
int get_weights(int tid){

    int res = EXIT_FAILURE;
    int no_weights_fit_cond; // condition for the failure of fill_bag

    res = pthread_mutex_lock( &gym_mtx );
    handle_error(res, errno, "phtread_mutex_lock ");

    set_train_status(GET_WEIGHTS, tid);
    printf("Tid %d GET_WEIGHTS --- ", tid  );
    display_status();

    no_weights_fit_cond = fill_bag( &bags[tid] );
    while( no_weights_fit_cond && !is_quit ){
        res = pthread_cond_wait( &gym_cv, &gym_mtx);
        handle_error(res, errno, "phtread_cond_wait ");
        no_weights_fit_cond = fill_bag( &bags[tid] );
    }

    set_train_status(WORKOUT, tid);
    printf("Tid %d WORKOUT     --- ", tid  );
    display_status();

    res = pthread_mutex_unlock( &gym_mtx );
    handle_error(res, errno, "phtread_mutex_unlock ");
    return EXIT_SUCCESS;
}

/* -------------------------------------------------------------PUT_WEIGHTS */
int put_weights(int tid){
    int res= EXIT_FAILURE;
    res = pthread_mutex_lock( &gym_mtx );
    handle_error( res, errno,"pthread_mutex_lock ");

    set_train_status(PUT_WEIGHTS, tid);
    printf("Tid %d PUT_WEIGHTS --- ", tid  );
    display_status();

    flush_bag(&bags[tid]);

    res = pthread_cond_broadcast( &gym_cv );
    handle_error( res, errno,"pthread_cond_broadcast ");

    set_train_status(REST, tid);
    printf("Tid %d REST        --- ", tid  );
    display_status();

    res = pthread_mutex_unlock( &gym_mtx );
    handle_error( res, errno,"pthread_mutex_unlock ");
    return EXIT_SUCCESS;
}
  \end{ccode}

\subsubsection{Befehl wird zur höflichen Bitte}
  Die Befehle werden nach dem Entschlüssen (siehe bei Sonstiges), in ein
  globales Feld gelegt. Der Thread prüft in seinem \texttt{WORK\_OUT}
  oder \texttt{REST} Zustand (in den Zählerschleifen), ob es einen Befehl
  gibt und blockiert sich ggf. selber mit seinem Semaphoren oder überspringt
  die Zählerschleife oder setzt zudem seinen Befehlszustand auf \texttt{QUIT}
  und terminiert. Einzige Ausnahme von dieser höflichen Form ist der
  Befehl zum Entblocken. Dort wird direkt der Semaphor um einen erhöht und
  somit der Thread wieder geweckt.


\subsection{Fehlerbehandlung}
Grundsätzlich terminiert die Funktion \texttt{handle\_error} das Programm. Also
beendet jeder fehlgeschlagene Bibliotheksfunktionsaufruf (sofern alle
abgegriffen werden) das Programm. Fehler, die wir selber definieren, wie das Eingeben
eines nicht akzeptierten Befehls, werden zu Kenntnis genommen aber 
weitestgehend toleriert. 
\subsection{Sonstiges}

Wie werden die Befehle ausgelesen:
\texttt{fgets} liefert einen String zurück, der leider auch das
  Zeilenumbruchzeichen von dem \texttt{NULL} einfügt.
  Bevor man das Befehlsmuster mit dem erhaltenen Befehlsstring vergleicht,
  wird der Zeilenumbruch mit \texttt{NULL} ersetzt. In kurzform so:

  \begin{ccode}
    fgets(cmd_s, NBUFF,stdin)
    cmd_s[strcspn(cmd_s, "\n")]= '\0';
    if (strcmp("q", cmd_s) ) ....
  \end{ccode}

  Die Threadnummer erhalten wir, indem man mit einer Zählerschleife alle
  Threadnummermöglichkeiten mit dem Characterwert der Null summiert und das
  Ergebnis mit dem ersten Character im Befehlsstring vergleicht.
  Nach erkannter Threadnummererkennung wird der zweite Character mit
  Befehlsmustern verglichen und zur richtigen Ausführung verzweigt.

  \begin{ccode}
    for(int i =0; NPHILO > i ; i++){
        int target = -1;
        if (cmd_s[0]== i + '0' ){
            target = i;
        }
    }
    if (0 <= target ){
        switch (cmd_s[1]){
            case b:
            ...
  \end{ccode}


\subsection{Kleine Anmerkungen aus dem Labortermin}
  Die Berechnung der Gewichte für die jeweilige Trainigseinheit erfolgt
  rekursiv mit folgenden Funktionen aus:
  \begin{ccode}
/*---------------------------------------------------------- FILL BAG ------*/
int fill_bag_recu( const int required_kg,
                   const int index,
                   Weight_bag_t * const bag){
    int remain_kg = required_kg;
    int success = EXIT_FAILURE;

    int qty = required_kg/ stock.weights[index].kg;
    if ( stock.weights[index].qty < qty ){
        qty = stock.weights[index].qty;
    }

    while ( 0 <= qty){

        remain_kg = required_kg - stock.weights[index].kg * qty;
        if ( 0 == remain_kg ){
            success = EXIT_SUCCESS;
        } else {
            if ( 0 < index ){
                success = fill_bag_recu(remain_kg, index - 1, bag);
            } else {
                success = EXIT_FAILURE;
            }
        }

        if ( EXIT_SUCCESS == success ){
            bag->weights[index].qty = qty;
            stock.weights[index].qty -= qty;
            stock.available_kg -= qty * stock.weights[index].kg;
            break;
        } else if ( EXIT_FAILURE == success ){
            qty--;
        }
    }
    return success;
}
/* ------------------------------------------------- fill_bag wrapper ----- */
int fill_bag(Weight_bag_t * bag){
    int success = EXIT_FAILURE;
    int required_kg= bag->required_kg;
    if (stock.available_kg >= required_kg && 0 < required_kg){
    int weights_len = NWEIGHTS_T;
        success = fill_bag_recu(required_kg, weights_len--, bag);
    }
    return success;
}
  \end{ccode}
Es wird dabei wie folgt vorgegangen:
\begin{itemize}
  \item Es werden von dem schwersten Gewichten, so viele vorgemerkt, sodass das
  benötigte Gewicht ich an die Null approximiert.
  \item In einer Whileschleife (solange die Anzahl der vorgemerkten Gewichte
  größer Null ist) wird getestet, ob diese Gewichte schon reichen, damit.
  \item Gewichte reichen, Rückgabevariable wird auf EXIT\_SUCCESS gesetzt;
  \item Gewichte reichen nicht, es wird, falls vorhanden, auf das nächst
  leichtere Gewicht mit einem rekursiven Aufruf zugegriffen. Dabei wird das
  benötigte Restgewicht mit übergeben. Der Rückgabewert von diesem Aufruf wird
  auf die Rückgabevariable gelegt:
  \item Falls kein leichteres Gewicht verfügbar ist, wird die Rückgabevariable
  auf EXIT\_FAILURE gesetzt.
  \item Die Rückgabevariable wird getestet:
  \item falls EXIT\_SUCCESS: Gewichte werden in den Beutel gepackt und aus dem
  Lager entfernt und es wird die Schleife verlassen, was zu einer Rückgabe des
  EXIT\_SUCCESS führt.
  \item falls EXIT\_FAILURE: Die Anzahl der Gewichte wird um ein reduziert.
  \item whileschleife fängt von vorne an, oder falls die Anzahl null ist,
  endet die Funktion und ein EXIT\_FAILURE wird zurückgegeben.

\end{itemize}

\subsubsection*{Stringmodifikation in gym.c}
Es wurde zudem diskutiert, das Anstelle von der Bibliotheksfunktion
\texttt{strcat} die etwas kontrolliertere \texttt{strncat} verwedent werden
sollte. Im Prinzip ist die \texttt{strncat} vorzuziehenl da man hier die Anzahl
der zu koncatinierenden Zeichen festlegen, was unerwünschtes überschreiben von
Speicherbereichen verhindern kann. Der Quellcode beinhaltet jedeoch noch die
andere Funktion, ohne Buffergrenze, da wir sichgestellt haben, dass es nicht
zu einem überlauf kommen kann. (Siehe die sehr komplizierten Defines der
Stringformate und Buffergrößen in \texttt{gym.h}.)

\subsubsection{Funktion quid\_philo in philo.c}
Wir sollten diese Funktion wegen diesem Schreibfehlers mittels Refactoring
zu \texttt{quit\_philo} ändern. Die Funktion wurde nirgendwo aufgerufen, sodass
sie auskommentiert ( wegen nachweislichen Zwecken ) wurde.

\end{document}
