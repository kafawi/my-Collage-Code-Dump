\documentclass[
   draft=false
  ,paper=a4
  ,twoside=false
  ,fontsize=11pt
  ,headsepline
  ,BCOR10mm
  ,DIV11
  ,parskip=full+
]{scrartcl} % copied from Thesis Template from HAW

\usepackage[ngerman,english]{babel}
\usepackage[T1]{fontenc}
\usepackage[utf8]{inputenc}

\usepackage[german,refpage]{nomencl}

\usepackage{float}
\usepackage{enumitem}
\usepackage{hyperref} % for a better experience

\hypersetup{
   colorlinks=true % if false - links get colored frames
  ,linkcolor=black % color of tex intern links
  ,urlcolor=blue   % color of url links
}

\usepackage{array}   % for \newcolumntype macro
\newcolumntype{L}{>{$}l<{$}} % math-mode version of "l" column type
\newcolumntype{R}{>{$}r<{$}} % math-mode version of "r" column type
\newcolumntype{C}{>{$}c<{$}} % math-mode version of "c" column type

%\usepackage{listing}
\usepackage{caption}
%\usepackage{xcolor}
%\definecolor{bg}{rgb}{0.60,0.95,0.95}
%using minted because of the hashtag in bash
\usepackage{minted}
% c listing
\newminted{c}{fontsize=\small
             ,fontfamily=tt
             ,linenos
             ,frame=single
             } % \begin{ccode} ... \end{ccode}
\newmintedfile{c}{fontsize=\small
                 ,fontfamily=tt
                 ,linenos
                 ,frame=single
                 ,autogobble
                 } % \cfile{}
% Makefile listing
\newminted{make}{fontsize=\small
             ,fontfamily=tt
             ,linenos
             ,frame=single
             }  % \begin{makecode} ... \end{makecode}
\newmintedfile{make}{fontsize=\small
                    ,fontfamily=tt
                    ,linenos
                    ,frame=single
                    ,autogobble
                    } % \makefile{}
\sloppy
\clubpenalty=10000
\widowpenalty=10000
\displaywidowpenalty=10000

\begin{document}

\selectlanguage{ngerman}
\usemintedstyle{emacs}
% ----------------------------------------------------------------------------
% Titel (erst nach \begin{document}, damit babel bereits voll aktiv ist:
\titlehead{Betriebsystene WS 2016 Praktikum 02}% optional
\subject{BS Praktikumsaufgabe 02}
\title{Thread Koordination: Die trainierenden PhilosophInnen}
\subtitle{Version 0.1 - Abgabe am 7. November 2016 16:00}
\author{Alexander Mendel \\ Karl-Fabian Witte}
\date{erstellt am \today}% sinnvoll
%\publishers{Platz für Betreuer o.\,ä.}% optional
% ----------------------------------------------------------------------------
\maketitle% verwendet die zuvor gemachte Angaben zur Gestaltung eines Titels
\begin{abstract}
  Zu Anfang wird ein Makefile zu einem gegebenen C-Datei
  erstellt und somit das Grundprinzip von (GNU)make gelernt.
  In der Hauptaufgabe wird die Synchronisation zwischen Threads geübt.
  Dabei wird unter anderem das Prinzip eines Monitors verwendet.
\end{abstract}
\tableofcontents
% ----------------------------------------------------------------------------
\flushleft
\section{Vorübung: Makefile}
Für den Entwurf wird das Makefile einfach hier reinkopiert.

\begin{makecode}
  EXE = critsect02

  CC = /usr/bin/gcc

  CFLAG  = -pthread
  CFLAG += -g
  CFLAG += -Wall

  LDFLAG = -lpthread

  SRC = critsect02.c

  # targets (all is default, because it is on top.)
  all: $(EXE)

  run: $(EXE)
  	./$(EXE)

  $(EXE): $(SRC)
  	$(CC) $(CFLAG) $(LDFLAG) -o $@ $<

  clean:
  	rm -f $(EXE)
\end{makecode}


\section{Thread-Koordination in der Muckibude}

In dieser Aufgabe soll die Thread-Koordination mit den Bibliotheken der
POSIX Threads und Semaphoren unter Linux geübt werden. Dafür schicken wir 5
Philosophen (Threads) in die Muckibude. Wie wir die Koordination der Gewichte
(Daten) zwischen den Philosophen realisieren, soll in den folgenden Abschnitten
erläutert werden.

\subsection{Überblick}
Dieser Abschnitt dient dazu, um sich schnell ein Bild machen zu können.

Das Programm wird aus folgenden Quellcodedatein zusammengestellt:
\begin{enumerate}
  \item[parameter.h] Hier sind alle Konstanten für die Implementierung
    definiert.
  \item[gym.c] Realisierung eines Monitors mit zwei Monitorfunktionen und den
    (empfindlichen) Daten: den Gewichten. Zudem weitere Hilfsfunktionen.
  \item[gym.h] Das Interface für Funktionen und Typedefinitionen.
  \item[philo.c] Die Threadfunktion, welche die Philosophen zu einem Zyklus
    des Gewichte Holens, Hebens, Weglegens, Ausruhens treibt.Zudem Deklaration
    des globalen Behlsfeldes.
  \item[philo.h] Das Interface und die externe Bekanntmachung des globalen
   Befehlsfeldes.
  \item[global.c] Die Funktion \texttt{handle\_error} wird hier implementiert.
    Damit wird der Rückgabewert von einer Bibliotheksfunktion ausgewertet,
    bei einem Fehler die entsprechende Meldung ausgegeben und das Programm
    beendet.
  \item[global.h] Das Interface für \texttt{handle\_error}. Zudem werden hier
    \texttt{TRUE} und \texttt{FALSE} definiert.
  \item[main.c] Die Mainfunktion mit weiteren Hilfsfunktionen. Der Mainthread
    koordiniert die Initialisierung und Zerstörung der anderen Threads und
    deren Objekte. Zudem steuert der Mainthread die interaktive Befehlannahme
    über die Konsole.
\end{enumerate}

Es sollen kurz die Prototypen der wichtigesten Funktionen aufgelistet werden:
\begin{ccode}
//     GYM
void init_gym_obj();  // init the stock, bags, pthread_mutex and pthread_cond

void free_gym_obj();  // destroys the pthread_mutex and pthread_cond
void free_gym_threads(); //  set is_quit = TRUE and pthread_cond_broadcast

int get_weights( int tid );  // monitor func
int put_weights( int tid );  // monitor func
int display_status( void );  // is called in monitor functions

int fill_bag( weight_bag_t * bag);  // get the weights form stock into bag
void flush_bag(weight_bag_t * bag); // put the weights back in stock

//     PHILO
void init_philo_obj(void);  // init philostructur, semaphore
void free_philo_obj(void);  // destroy the semaphore

void * philothread (void * ptr_tid); // thread function, no return, param tid

void set_train_status(train_state_t state, int tid); // used by monitor func

void unblock_philo(int tid);  // used to unblock the thread

//     GLOBAL
void handle_error(            // an universal tool
int result, int errnum, char const *mssge
);

\end{ccode}


Und hier der Parameterheader
\begin{ccode}
  #define NPHILO      5
  //#define PHILO_NAMES { "Anne", "Bernd", "Clara", "Dirk", "Emma" }
  #define PHILO_KG    {      6,       8,      12,     12,     14 }

  #define NWEIGHTS_T 3
  #define WEIGHT_KG  { 2, 3, 5 }
  #define WEIGHT_QTY { 4, 4, 5 }

  #define REST_LOOP   1000000000
  #define WORKOUT_LOOP 500000000
\end{ccode}


\subsubsection{Programmablauf}
\textbf{Der Mainthread} erzeugt bzw. initialisiert alle Objekte. Dann werden
die Pthreads (Philosophen) gestarted. Danach ist der Mainthread in der
Kontrollerschleife. In dieser Endlosschleife wartet der
Mainthread auf eine Eingabe aus der \texttt{stdin} und überprüft diesen
auf die Befehle:
\begin{enumerate}
  \item[q oder Q] Beendet das Programm sauber.
  \item[\texttt{tid}b] Blockier Thread Nummer tid.
  \item[\texttt{tid}u] Befreit Thread Nummer tid.
  \item[\texttt{tid}p] Thread Nummer tid überspringt die REST und WORK\_OUT
   Schleifen.
\end{enumerate}

Ein akzeptierte Befehl wird entschlüsselt und
bearbeitet. Wenn der Befehl zum Schließen des Programms erkannt wurde, werden
alle Philosophethreads gebeten, sich zu beenden (über das globale
Befehlsfeld). Alle blockierten Threads werden befreit, indem man die
Blockierungsobjekte freigibt. Der Mainthread wartet nun dadrauf, dass alle
Threads terminierten und zerstört alle erstellten Objekte. Erst dann beendet
sich der Mainthread und terminiert somit das Programm.

\textbf{Einem Philothread} wird die Nummer (\texttt{tid}) mitgegeben, mit
welcher er auf die für ihn wichtigen Felder zugreifen und diese manipulieren kann.
Er durchläuft in einer Endlosschleife, in der er vier Funktionen pro Schleife
ausführt. Zwei davon sind fast leere Zählerschleifen, in den er auf die
Befehle in dem globalen Array reagiert. Die anderen beiden Funktionen werden
durch das Monitormodell realisiert, um die Konsistenz der Daten (Gewichte)
wahren zu können. In diesen beiden Funktionen ignoriert der Philothread zunächst die
Befehle des Mainthreads. In den Funktionen des Monitors wird beim Eintreten in
und beim Austreten aus dem geschützten Bereich innerhalb diesem der Status auf
der Console ausgegeben.

\subsection{Sichtbarkeit der Objekte}
  Da es sich um eine statische Anzahl an Threads und Objekten handelt, konnen
  die Objekte ebenfalls statisch sein. Unsere Synchronisationsobjekte haben
  zudem Dateisichtbarkeit. Somit wird für jede C-Datei eine Initialisierungs-
  und eine Zerstörungsfunktion implementiert, welche dann von dem Mainthread
  aufgerufen wird.
  Das globale (zwischen mehreren C-Datein sichtbar) Befehlsfeld ist von
  main.c und von philo.c aus sichtbar.

  Wir haben uns gegen dynamischen HEAP-Speicher entschieden, das dies keinen
  Sinn ergebe, da kein Mehrwert für uns zu erkennen war.

\subsubsection{Struckturen der Objekte}

Gerade die zu schützenden Objekt sind hier von großer Bedeutung, die Gewichte.
Dabei verwaltet gym.c alle weight Objekte, auch die der Philosophen.
Beim erstenblick ist das zunächst komisch, warum der Philosoph selber nicht
weiß, wie viel er für sein Training an kg bekommt. Die Muckibude brauch als
einzige diesen Wert, um die Gewichte aus dem Lager (\texttt{stock}) in die für
den Philosophen bereitgestellten Gewichtetasche zu füllen (\texttt{bag});

Deswegen hat die Muckibude folgende Struckturen im statischen privaten
Sichtbarbeitsfeld:
\begin{ccode}
  typedef struct {
      int kg;        // type
      int qty;       // quantity
  } weight_t;

  struct{
      weight_t weights[ NWEIGHTS_T ]; // slots
      int total_kg;                   // total amount of all weights in the gym
      int available_kg;               // amount what is left
  } stock;

  typedef struct {
      weight_t weights [ NWEIGHTS_T ];   // slots
      int required_kg;                   // required for a effective trainigs
  } weight_bag_t;

  weight_bag_t bags[NPHILO];             // for every philo one bag

  pthread_mutex_t gym_mtx;
  pthread_cond_t gym_cv;
\end{ccode}

Die Philosphen haben primär die Information, in welchem Status sie sich grade
befinden:
\begin{ccode}
typedef enum {
    QUIT=20,
    BLOCKED,
    NORMAL
} cmd_state_t;

typedef enum {
    GET_WEIGHTS=10,
    WORKOUT,
    PUT_WEIGHTS,
    REST
} train_state_t;

typedef struct {
    sem_t mtx;
    cmd_state_t cmd_state;
    train_state_t train_state;
    int tid;
} philo_t;

extend char cmd_arr[];
\end{ccode}

\subsection{Die Muckibude als Monitor}
  Die Muckibude \texttt{gym.c} hat als Synchronisationsobjecte einen POSIX
  Mutex \texttt{gym\_mtx} und eine POSIX Bedingungsvariable \texttt{gym\_cv}.
  Beim Eintreten und Verlassen des geschützten Bereich, wird jeweils die
  Der Trainierstatus des Philothreates entsprechend geänder und
  \texttt{display\_status} aufgerufen. \texttt{display\_status} überprüft
  ob auch kein Gewicht verloren gegengen ist und gibt ggf eine Fehlermeldung
  aus.

  Nur die Threads, die auf die sich Gewichte holen wollen
  (\texttt{GET\_WEIGHTS}), können auf Grund der beschränkten verfügbaren Gewichte die
  erforderlichen Gewichte nicht holen. Somit warten
  sie auf der POSIX Bedingungsvariable in einer Whileschleife und geben den
  geschützen Bereich wieder frei. Wenn aber einige Gewicht von einem anderen
  Thread wieder zurückgelegt wurden (\texttt{PUT\_WEIGHTS}), wird vor dem
  Auftreten aus dem sicheren Bereich alle auf Bedingungsvariablen sitzenden
  Threads geweckt, welche dann wiederum prüfen können, ob sie diesmal die
  Gewichte kriegen können.

  Das folgende Listing ist ein Auszug aus unserem bisherigen Code. Es zeigt die
  beiden Monitorfunktionen. Die zweite Bedingung ist für die Freisetzung der
  blockierenden Threads, nachdem der Beendenbefehl erteilt wurde.

  \begin{ccode}
    int get_weights(int tid){
        int no_weights_fit_cond;
        int res=-1;
        res = pthread_mutex_lock( &gym_mtx );                 // MONITOR START
        handle_error( res, errno,"pthread_mutex_lock ");

        set_train_status(GET_WEIGHTS, tid);
        display_status();

        no_weights_fit_cond = fill_bag( &bags[tid] );
        while( no_weights_fit_cond  || !is_quit ){
            res = pthread_cond_wait( &gym_cv, &gym_mtx);
            handle_error( res, errno,"pthread_cond_wait ");
            get_weight_cond = fill_bag( &bags[tid] );
        }

        set_train_status(WORKOUT, tid);
        display_status();
        res = pthread_mutex_unlock( &gym_mtx );                  // MONITO END
        handle_error( res, errno,"pthread_mutex_unlock ");
        return 0;
    }
    //------------------------------------------------------------------------
    int put_weights(int tid){
        int res=-1;
        res = pthread_mutex_lock( &gym_mtx );                 // MONITOR START
        handle_error( res, errno,"pthread_mutex_lock ");

        set_train_status(PUT_WEIGHTS, tid);
        display_status();

        flush_bag(&bags[tid]);

        res = pthread_cond_broadcast( &gym_cv );     // wakes all cond waiters
        handle_error( res, errno,"pthread_cond_signal ");
        set_train_status(REST, tid);
        display_status();

        res = pthread_mutex_unlock( &gym_mtx );                 // MONITOR END
        handle_error( res, errno,"pthread_mutex_unlock ");
        return 0;
    }
  \end{ccode}

\subsubsection{Befehl wird zur höflichen Bitte}
  Die Befehle werden nach dem Entschlüssen (siehe bei Sonstiges), in ein
  globales Feld gelegt. Der Thread prüft in seinenem \texttt{WORK\_OUT}
  oder \texttt{REST} Zustand (also in den Zählerschleifen), ob es einen Befehl
  gibt und blokiert sich ggf. selber mit seinem Semaphoren oder überspringt
  die Zählerschleife oder setzt zudem seinen Befehls Zustand auf \texttt{QUIT}
  und terminiert. Einzige Ausnahme von dieser höflichen Form ist der
  Befehl zum Entblocken. Dort wird direkt der Semaphor um einen erhöht und
  somit der Thread wieder geweckt.


\subsection{Fehlerbehandlung}
Grundsätzlich terminiert die Funktion \texttt{handle\_error} das Programm. Also
beendet jeder fehlgeschlagene Bibliotheksfunktionsaufruf (sofern alle
abgegriffen werden) das Programm. Fehler, die wir selber definieren, wie das
Abhandenkommen von Gewichten, werden nur zur Kenntnis genommen. Auch ein
ungültiger Befehl gibt kurz eine Meldung aus lässt das Programm weiter laufen.

\subsection{Sonstiges}

Wie werden die Befehle ausgelesen:
\texttt{fgets} liefert einen String zurück, der leider auch das
  Zeilenumbruchzeichen von dem \texttt{NULL} einfügt.
  Bevor man das Befehlsmuster mit dem erhaltenen Befehlsstring vergleicht,
  wird der Zeilenumbruch mit \texttt{NULL} ersetzt. In kurzform so:

  \begin{ccode}
    fgets(cmd_s, NBUFF,stdin)
    cmd_s[strcspn(cmd_s, "\n")]= '\0';
    if (strcmp("q", cmd_s) ) ....
  \end{ccode}

  Die Threadnummer erhählten wir, indem man mit einer Zählerschleife alle
  Threadnummermöglichkeiten mit dem Characterwert der Null summiert und das
  Ergebnis mit dem ersten Character im Befehlsstring vergleicht.
  Nach erkannter Threadnummererkennung wird der zweite Character mit
  Befehlsmustern verglichen und zur richtigen Ausführung verzweigt.

  \begin{ccode}
    for(int i =0; NPHILO > i ; i++){
        int target = -1;
        if (cmd_s[0]== i + '0' ){
            target = i;
        }
    }
    if (0 <= target ){
        switch (cmd_s[1]){
            case b:
            ...
  \end{ccode}



\end{document}
