\documentclass[
   draft=false
  ,paper=a4
  ,twoside=false
  ,fontsize=11pt
  ,headsepline
  ,BCOR10mm
  ,DIV11
  ,parskip=full+
]{scrartcl} % copied from Thesis Template from HAW

\usepackage[ngerman,english]{babel}
\usepackage[T1]{fontenc}
\usepackage[utf8]{inputenc}

\usepackage[german,refpage]{nomencl}

\usepackage{float}
\usepackage{enumitem}
\usepackage{hyperref} % for a better experience

\hypersetup{
   colorlinks=true % if false - links get colored frames
  ,linkcolor=black % color of tex intern links
  ,urlcolor=blue   % color of url links
}

\usepackage{array}   % for \newcolumntype macro
\newcolumntype{L}{>{$}l<{$}} % math-mode version of "l" column type
\newcolumntype{R}{>{$}r<{$}} % math-mode version of "r" column type
\newcolumntype{C}{>{$}c<{$}} % math-mode version of "c" column type

%\usepackage{listing}
\usepackage{caption}
%\usepackage{xcolor}
%\definecolor{bg}{rgb}{0.60,0.95,0.95}
%using minted because of the hashtag in bash
\usepackage{minted}
% c listing
\newminted{c}{fontsize=\small
             ,fontfamily=tt
             ,linenos
             ,frame=single
             } % \begin{ccode} ... \end{ccode}
\newmintedfile{c}{fontsize=\small
                 ,fontfamily=tt
                 ,linenos
                 ,frame=single
                 ,autogobble
                 } % \cfile{}
% Makefile listing
\newminted{make}{fontsize=\small
             ,fontfamily=tt
             ,linenos
             ,frame=single
             }  % \begin{makecode} ... \end{makecode}
\newmintedfile{make}{fontsize=\small
                    ,fontfamily=tt
                    ,linenos
                    ,frame=single
                    ,autogobble
                    } % \makefile{}
\sloppy
\clubpenalty=10000
\widowpenalty=10000
\displaywidowpenalty=10000

\begin{document}

\selectlanguage{ngerman}
\usemintedstyle{emacs}
% ----------------------------------------------------------------------------
% Titel (erst nach \begin{document}, damit babel bereits voll aktiv ist:
\titlehead{-- titel head --}% optional
\subject{-- subject --}
\title{-- title --}
\subtitle{-- subtitle --}
\author{-- author --}
\date{erstellt am \today}% sinnvoll
%\publishers{Platz für Betreuer o.\,ä.}% optional
% ----------------------------------------------------------------------------
\maketitle% verwendet die zuvor gemachte Angaben zur Gestaltung eines Titels
\begin{abstract}
  ABSTRACT ... ABSTRACT ...
\end{abstract}
\tableofcontents
% ----------------------------------------------------------------------------
\flushleft
\section{-- section 01 --}
ONE ONE SECTION ONE ONE

\section{Quellcode}
Der komplette Quellcode zu den Aufgaben sowie zu diesem Dokument kann im
Repository unter folgender URL betrachtet werden:\newline
\url{https://bitbucket.org/bibutNkafawi/irgendwas.com} \newline
oder einfach:

\begin{minted}[fontsize=\small
              ,fontfamily=tt
              ,frame=lines]{console}
$  git clone https://usr@bitbucket.org/bibutNkafawi/irgendwas.git
\end{minted}
\end{document}
