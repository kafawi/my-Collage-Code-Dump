\documentclass[
   draft=false
  ,paper=a4
  ,twoside=false
  ,fontsize=11pt
  ,headsepline
  ,BCOR10mm
  ,DIV11
  ,parskip=full+
]{scrartcl} % copied from Thesis Template from HAW

\usepackage[ngerman,english]{babel}
\usepackage[T1]{fontenc}
\usepackage[utf8]{inputenc}

\usepackage[german,refpage]{nomencl}

\usepackage{float}
\usepackage{enumitem}
\usepackage{hyperref} % for a better experience

\hypersetup{
   colorlinks=true % if false - links get colored frames
  ,linkcolor=black % color of tex intern links
  ,urlcolor=blue   % color of url links
}

\usepackage{array}   % for \newcolumntype macro
\newcolumntype{L}{>{$}l<{$}} % math-mode version of "l" column type
\newcolumntype{R}{>{$}r<{$}} % math-mode version of "r" column type
\newcolumntype{C}{>{$}c<{$}} % math-mode version of "c" column type

%\usepackage{listing}
\usepackage{caption}
%\usepackage{xcolor}
%\definecolor{bg}{rgb}{0.60,0.95,0.95}
%using minted because of the hashtag in bash
\usepackage{minted}
\usepackage{mdframed}
%\newfloat{program}{tbp}{lop}%
\sloppy

\clubpenalty=10000
\widowpenalty=10000
\displaywidowpenalty=10000


\begin{document}

\selectlanguage{ngerman}
\usemintedstyle{emacs}
% ----------------------------------------------------------------------------
% Titel (erst nach \begin{document}, damit babel bereits voll aktiv ist:
\titlehead{Betriebsysteme WS 2016 Praktikum 01 Protokoll}% optional
\subject{BS Praktikumsaufgabe 01}
\title{Der Umgang mit Linux}
\subtitle{Version 1.0 - Abgabe am 25. Oktober 2016}
\author{Alexander Mendel  Karl-Fabian Witte}
\date{erstellt am \today}% sinnvoll
%\publishers{Platz für Betreuer o.\,ä.}% optional
% ----------------------------------------------------------------------------
\maketitle% verwendet die zuvor gemachte Angaben zur Gestaltung eines Titels
\begin{abstract}
  In diesem Praktikum soll der Umgang mit Linux geübt werden.
  Dafür werden Grundprinzipien und Aufbau der Linuxschnittstellen geübt.
  Zum Schluss soll die \texttt{bash} noch einmal genauer mit einem Skript
  gelernt werden.
\end{abstract}
\tableofcontents
% ----------------------------------------------------------------------------
\flushleft
\section{Die Macht der Kommandozeile}
Es wird kurz der Umgang mit dem Terminal geübt. Es werden hier für die
Aufgeben wichtigen Informationen dokumentiert. Das mitgelaufene Logbuch
\texttt{typescript}, welches mittels \texttt{script} gestartet wurde, ist im
Reposetory zu finden (siehe Abschnitt 3). Zudem werden einige fundamentalen
Kenntnisse für Linuxnutzer mit auf den Weg gegeben.

\begin{enumerate}
  \item \textbf{Was ist \emph{Tab-Expansion} und was nützt Ihnen das bei der
  Arbeit mit der Kommandozeile?}\newline
  Die \emph{Tab-Expansion} vervollständigt das angefangene Befehlswort (oder
  auch Verzeichnis- und Dateinamen) in der Shell-Kommandozeile. Wenn im
  Terminal \textbf{<Tab>} gedrückt wird, wird der angefangene String bis zur
  Mehrdeutigkeit vervollständigt und beim erneuten Drücken der \textbf{<Tab>}
  werden die Möglichkeiten aufgezeit. Das folgende Beispiel verdeutlich es:
  \begin{minted}{bash}
  $ ls -1
  xy.pdf
  xy.ps
  $ acroread x               <- <Tab> druecken
  $ acroread xy.p            <- nochmal <Tab> druecken
  xy.pdf xy.ps
  $ acroread xy.p
  \end{minted}

  \item \textbf{Was erhalten Sie beim Drücken der Tastenkombination
  \textbf{<Alt><.>}?}\newline
  \textbf{<Alt><.>} gibt das letzte Argument des letzten Befehls wieder.
  Bein erneuten drücken wird das letzte Argument der vorletzten Befehls
  an die Coursorposition gesetzt, und so weiter.

  \item \textbf{Geben Sie das Verzeichnis nach \emph{Erweiterung} sortiert aus.
  }
  \begin{minted}{bash}
  $ ls -X
  \end{minted}

  \item \textbf{Geben Sie das Verzeichnis nach \emph{Modifikationszeit}
  sortiert aus.}
  \begin{minted}{bash}
  $ ls -t
  \end{minted}

  \item \textbf{Kehren Sie für beide Sortiervarianten die \emph{Reihenfolge}
  um.}
  \begin{minted}{bash}
  $ ls -Xr
  $ ls -tr
  \end{minted}

  \item \textbf{Geben Sie das Verzeichnis nach \emph{rekursiv} d.h. mit allen
  Unterverzeichnissen aus.}
  \begin{minted}{bash}
  $ ls -R
  \end{minted}

  \item \textbf{Erläutern Sie die Operatioen von dem sort-Befehl
  \texttt{ls -l | sort -rnk5}.}
  \begin{minted}{bash}
  $ ls -l | sort -rnk5
------------------------
  -r  : umgekehrte Reihenfolge
  -n  : numerische Wertsortierung 0 < 1 < 2 ...
  -k5 : die 5te Spalte (hier mit ls -l -> Dateigroesse)
  \end{minted}
  Es wird also nach Dateingröße absteigend sortiert.

  \item \textbf{Dokumentieren Sie mit \texttt{ls -l } das Resultat Ihrer
  Aktion.}
  \begin{minted}{bash}
  bs@linux-8b19:~/Bs_Prak> ls -l
  insgesamt 16
  lrwxrwxrwx 1 bs users    9 17. Okt 03:36 ltext2.txt -> text2.txt
  lrwxrwxrwx 1 bs users    9 17. Okt 03:37 ltext3.txt -> text3.txt
  -rw-r--r-- 1 bs users 3976 17. Okt 03:30 my_listing.txt
  -rw-r--r-- 1 bs users   28 17. Okt 03:33 text04.txt
  -rw-r--r-- 1 bs users   28 17. Okt 03:33 text2.txt
  -rw-r--r-- 1 bs users   28 17. Okt 03:33 text3.txt
  \end{minted}

  \item \textbf{Editieren Sie \texttt{ltext3.txt}: Wie verändert sich
  \texttt{text3.txt}?}\newline
  Die Änderung ist von \texttt{ltext3.txt} ist auch in \texttt{text3.txt}
  gespeichert worden.

  \item \textbf{Was passiert, wenn Sie \texttt{ltext2.txt} löschen?}\newline
  Die Beseitigung von \texttt{ltext2.txt} hat keine Auswirkung auf text2.txt

  \item \textbf{Was passiert, wenn Sie \texttt{text3.txt} löschen?}\newline
  Die Datei \texttt{ltext3.txt} ist ein symbolische Link, welcher wie eine
  Windows gewöhnliche Verknüpfung funtioniert: Er verzweigt nur auf die
  Originaldatei und enthält sonst keine Information. Da die Originaldatei nicht
  mehr existiert, funtioniert \texttt{ltext3.txt} auch nicht mehr.
  Anders wäre es mit einem harten Link (\texttt{ln text3.txt htext3.txt}).
  Dieser verweist auf die physikalische Adresse, wie es die Originalsdatei auch
  nur tut, und ist somit, von außen gesehen, eine synchronisierte Kopie.

  \item \textbf{Demonstrieren Sie die Platzhalterzeichen mit eigenen
  Beispielen.}
  \begin{minted}{bash}
  bs@linux-8b19:~/Bs_Prak> ls
  ltext3.txt my_listing.txt text04.txt text2.txt
  text04.txt text2.txt
  bs@linux-8b19:~/Bs_Prak> ls *ext?.txt
  ltext3.txt text2.txt
  bs@linux-8b19:~/Bs_Prak> ls [ml]*
  ltext3.txt my_listing.txt
  bs@linux-8b19:~/Bs_Prak> ls [k-n]*
  ltext3.txt my_listing.txt
  \end{minted}


  \item \textbf{Finden Sie heraus, was die Wirkung der Zeichen \texttt{\$},
  \texttt{\^{}} und \texttt{\textbackslash <} ist. Warum musste der
  Suchausdruck im letzten Beispiel in Anführungszeichen (\emph{quotes})
  gesetzt werden?}
  \begin{minted}{bash}
  $ ls -l /etc/ | grep "\<fs"
  # Es werden alle Zeilen ausgegeben, in denen Namen/Wörter vorkommen,
  # die mit "fs" beginnen.

  $ ls -1 /etc/ | grep ^fs
  # Es werden alle Zeilen ausgegeben,  mit "fs" beginnen.

  $ ls -l /etc/ | grep fs$
  # Es werden alle Zeilen ausgegeben, die mit fs enden.
  \end{minted}

  \item \textbf{Geben Sie alle Prozesse aus, deren Kommandozeile mit k
  beginnt.}
  \begin{minted}{bash}
  ps aux | grep ^k
  # oder wenn man de Befehle will
  # und nicht einen usernamen hat der mit k anfaengt
  ps aux | grep "\<fs"
  \end{minted}
\end{enumerate}
\clearpage

\section{Vorgänge automatisieren: Shellskripte}

Nun soll die Macht eines Shellskriptes kennen gelernt werden. Dafür wird erst
ein gegebenes Skript analysiert. Danach soll dann eins selber geschrieben
werden, was die Anzahl der Datein in einem Verzeichnis zurück gibt. Dabei wird
die \texttt{bash} verwendet und gehofft, dass der Interpreter jede Zeile auch
übersetzten kann.

\begin{enumerate}
  \item \textbf{Was tut das oben angegebene Shellskript?}\newline
  Es fragt nach den Anwender an seinem Namen und und gibt dann in einem
  \texttt{here file} eine schmucke ASCII-Begrüßung auf dem Terminal aus.
  Zudem kann man mit dem Flag \texttt{-h} oder \texttt{--help} eine
  Kurzanleitung sich ausgeben lassen. Alle weiteren Argumente werden nicht
  geduldet.

  \item \textbf{Wie bekommen Sie heraus, welche Version des C-Compilers
  \texttt{gcc} auf Ihrer virtuellen Maschine installiert ist?}\newline
  \texttt{gcc} ist ein ganz besonders gesittets Programm:
  \begin{minted}{bash}
  $ gcc --version
  \end{minted}

  \item \textbf{Schreiben Sie ein Shellskript \texttt{filecount.sh}, das die
  Anzahl von Dateien der als Option spezifizierten Typen ausgibt. Mit der
  Option \texttt{-h} oder \texttt{--help} soll es diesen Hilfetext ausgeben und
  damit gleichzeitig erklären, was genau Sie implementieren sollen.}
  \begin{minted}{bash}
cat<<EOF
  usage :
    $0 [ OPTIONS ] DIRECTORY

      Print the number of files in DIRECTORY
      of types given in OPTIONS

      Default: Print numbers of regular files
      in current directory

  OPTIONS :
  FILE type options:
    -f --regular-file  regular files
    -l --symlink       symbolic links
    -D --device        character and block devices
    -d --directory     directories
    -a --all           Include hidden files

  Other options:
    -h --help          Display this text
    -e --echo          Print selected files
    -v --verbose       print debugging messages
    -V --version       print Versionsnumber
EOF
  \end{minted}
Grundidee:
\begin{minted}{bash}
ls -l | egrep ^$TYPE | wc -l
# for echo
ls -l | egrep ^$TYPE | tee ...
\end{minted}

\end{enumerate}

\subsection{filecount.sh - der Entwurf}
Die Grundidee wurde uns ja schon gegeben. Beim Aufruf von \texttt{ls -l} ist
der erste Character in einer Zeile ein Bezeichner \texttt{\$TYPE} für den
\texttt{file type}. Diese Zeilen ziehen wir mit \texttt{egrep \^{}\$TYPE}
heraus und lassen anschließend die Zeilen mit \texttt{wc -l} zählen.

Die Optionsflag werden mit Hilfe der Funtion \texttt{getopt} herausgefiltert
und in einer while-Schleife eingebetteten Fallunterscheidung (\texttt{case})
die entsprechenden Optionen in dieser verarbeitet. Dabei ermöglicht
\texttt{getopt} das Zusammenfassen der sogenannten \emph{short options} ohne
diese Strings explizit in der Fallunterscheidung mit jeweils einem Fall zu
unterscheiden.

Die \texttt{file type opions} werden so behandelt, dass wir jeweils den
\texttt{file type character} hinten an den zu Anfang leeren \texttt{TYPES}-
String anhängen. Die Optionen \texttt{--help} und \texttt{--version} geben
einen entsprechenden Text im Terminal aus und beenden das Skript. Für
\texttt{--verbose} und \texttt{--echo} werden Flags gesetzt, die mit If-Blöcken
entsprechende Extrafunktionen aufrufen. Beipiel:
\begin{minted}{bash}
if [ VERBOSE -eq 1 ]; then
  echo "verbose mode activ"
fi
\end{minted}

Wir haben uns entschlossen haben, dass die type-flags sich nicht gegenseitig
ausschließen und man so eine Komposition an unteschiedlichsten Typen zählen
lassen kann, werden wird der String \texttt{TYPES} wie folgt nach den Regeln
der Regulären Ausdrücke verweden:
\begin{minted}{bash}
ls -l | egrep ^[$TYPES] | wc -l
\end{minted}

Wenn das all-Flag gesetzt wird, wird in zum einen in den \texttt{TYPES}-String
nicht angehängt, sondern einfach alle Möglichkeiten der \emph{file types}
geschrieben. (siehe \texttt{info ls}) Zudem, um auch versteckte Datein anzeigen
zu lassen, müssen die ls-Option mit \texttt{-A} erweitert werden. Somit haben
wir für diese Flags auch eine Variable \texttt{LSFLAGS} erstellt, um diese
bei \texttt{-a} entsprechend zu ändern. (\texttt{ls -A} is almost-all ist,
welche die hidden files darstellt, jedoch das current und parent directory
nicht mitauflistet.)

Sind alle Optionen angearbeitet, bleibt Dank \texttt{getopt} das nicht-Options-
Argument stehen. Dieses fragen wir zunächst ab, ob es überhaupt ein gibt. Wenn
keins übergeben wurde, bleibt die Variable \texttt{DIRECTORY} auf dem
Defaultwert \texttt{'.'}. Wenn es ein Argument gibt, so wird getestet, ob es
sich auch um ein \texttt{directory} handle. Wenn ja, wird dieses in
\texttt{DIRECTORY} gespeichert. Ansonsten wird eine Fehlermeldung ausgegeben.
Es wird nur ein directory akzeptiert, sodass ein weiters Argument auch zum
Abbruch führt. Ist nichts in den leeren String \texttt{TYPES} bei der
Falunterscheidungsschleife geschrieben worden, so wird dieser mit dem default
für reguläre Files \texttt{char '-'} gefüllt. Zum Schluss wird die eigendliche
Funktion mit dem Aufruf \texttt{calc} (eigene Funktion dieses Skripts)
gestartet.

Wir haben uns entschlossen, die Funktion \texttt{tee} anders zu verwenden: als
einfaches entfärbendes Rohr, anstatt eines T-Stücks. Die Alternative dazu wird
hier einmal vereinfacht dargestellt:
\begin{minted}{bash}
if [ $ECHO -eq 1 ]; then                    # echo mode
  ls $LSFLAG | egrep ^[$TYPES] | tee | wc -l
else                                        # no echo mode
  ls $LSFLAG | egrep ^[$TYPES] | wc -l
fi
\end{minted}

\subsection{filecount.sh - der Code}
\inputminted[fontsize=\small
            ,fontfamily=tt
            ,linenos
            ,frame=single
            ]{bash}{filecount.sh}
\captionof{listing}{filecount.sh : das komplette script}

\section{Quellcode}

In dem angehängten Verzeichnis ist das Shellskript \texttt{filecount.sh} und
die die Logdatei \texttt{typescript}.
Der komplette Quellcode zu den Aufgaben sowie zu diesem Dokument kann im
Repository unter folgender URL betrachtet werden:\newline
\url{https://bitbucket.org/bibutNkafawi/a01_der_umgang_mit_linux}

\end{document}
