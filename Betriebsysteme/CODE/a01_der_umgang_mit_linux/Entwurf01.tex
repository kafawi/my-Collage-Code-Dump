\documentclass[
   draft=false
  ,paper=a4
  ,twoside=false
  ,fontsize=11pt
  ,headsepline
  ,BCOR10mm
  ,DIV11
]{scrartcl} % copied from Thesis Template from HAW

\usepackage[ngerman,english]{babel}
\usepackage[T1]{fontenc}
\usepackage[utf8]{inputenc}

\usepackage[german,refpage]{nomencl}

\usepackage{float}
\usepackage{enumitem}

\usepackage{hyperref} % for a better experience
\hypersetup{
   colorlinks=true % if false - links get colored frames
  ,linkcolor=black % color of tex intern links
  ,urlcolor=blue   % color of url links
}

\usepackage{array}   % for \newcolumntype macro
\newcolumntype{L}{>{$}l<{$}} % math-mode version of "l" column type
\newcolumntype{R}{>{$}r<{$}} % math-mode version of "r" column type
\newcolumntype{C}{>{$}c<{$}} % math-mode version of "c" column type

%using minted because of the hashtag in bash
\usepackage{minted}


\begin{document}
\selectlanguage{ngerman}
\usemintedstyle{emacs}
% ----------------------------------------------------------------------------
% Titel (erst nach \begin{document}, damit babel bereits voll aktiv ist:
\titlehead{Betriebsysteme WS 2016 Praktikum 01 }% optional
\subject{BS Praktikumsaufgabe 01}% optional
\title{Der Umgang mit Linux}% obligatorisch
\subtitle{Version 0.1 - Abgabe am 17. Oktober 2016 16:00}% optional
\author{Alexander Mendel \\ Karl-Fabian Witte}% obligatorisch
\date{erstellt am \today}% sinnvoll
%\publishers{Platz für Betreuer o.\,ä.}% optional
% ----------------------------------------------------------------------------
\maketitle% verwendet die zuvor gemachte Angaben zur Gestaltung eines Titels
\begin{abstract}
  In deisem Praktikum soll der Umgang mit Linux geübt werden.
  Dafür werden grundprinzipien und Aufbau der Linuxschnittstelle geübt.
  Zum schluss soll die \texttt{bash} noch einmal genauer mit einem skript
  gelernt werden.  \end{abstract}
\tableofcontents
% ----------------------------------------------------------------------------

\section{Die Macht der Kommandozeile}
Es wird kurz der Umgang mit dem Terminal geübt. Es werden hier für die
Aufgeben wichtigen Informationen dokumentiert. Das mitgelaufene Logbuch,
welches mittels \texttt{script} gestartet wurde, ist im Reposetory zu finden
(siehe Abschnitt 3). Zudem werden einige fundamentalen Kenntnisse für
Linuxnutzer mit auf den Weg gegeben.

\begin{enumerate}
  \item \textbf{Was ist \emph{Tab-Expansion} und was nützt Ihnen das bei der
  Arbeit mit der Kommandozeile?} \\
  \emph{Tab-Expansion} vervollständigt das angefangene Befehlswort (oder auch
  Verzeichnis- und Dateinamen) in der Shell-Kommandozeile. Wenn im
  Terminal \textbf{<Tab>} gedrückt wird, wird der angefangene String bis zur
  Mehrdeutigkeit vervollständigt und beim erneuten Drücken der \textbf{<Tab>}
  werden die Möglichkeiten aufgezeit.\\
  Beispiel:
  \begin{minted}{bash}
  $ ls -1
  xy.pdf
  xy.ps
  $ acroread x               <- <Tab> druecken
  $ acroread xy.p            <- nochmal <Tab> druecken
  xy.pdf xy.ps
  $ acroread xy.p
  \end{minted}

  \item \textbf{Was erhalten Sie beim Drücken der Tastenkombination
  \textbf{<Alt><.>}?}\\
  \textbf{<Alt><.>} gibt das letzte Argument des letzten Befehls wieder.
  Bein erneuten drücken wird das letzte Argument der vorletzten Befehls
  an die Coursorposition gesetzt, und so weiter.

  \item \textbf{Geben Sie das Verzeichnis nach \emph{Erweiterung} sortiert aus.
  }\\
  \begin{minted}{bash}
  $ ls -X
  \end{minted}

  \item \textbf{Geben Sie das Verzeichnis nach \emph{Modifikationszeit}
  sortiert aus.}\\
  \begin{minted}{bash}
  $ ls -t
  \end{minted}

  \item \textbf{Kehren Sie für beide Sortiervarianten die \emph{Reihenfolge}
  um.}\\
  \begin{minted}{bash}
  $ ls -Xr
  $ ls -tr
  \end{minted}

  \item \textbf{Geben Sie das Verzeichnis nach \emph{rekursiv} d.h. mit allen
  Unterverzeichnissen aus.}\\
  \begin{minted}{bash}
  $ ls -R
  \end{minted}

  \item \textbf{Erläutern Sie die Operatioen von dem sort-Befehl
  \texttt{ls -l | sort -rnk5}.}\\
  \begin{minted}{bash}
  $ ls -l | sort -rnk5
------------------------
  -r  : umgekehrte Reihenfolge
  -n  : numerische Wertsortierung 0 < 1 < 2 ...
  -k5 : die 5te Spalte (hier mit ls -l -> Dateigroesse)
  \end{minted}
  Es wird also nach Dateingröße absteigend sortiert.

  \item \textbf{Dokumentieren Sie mit \texttt{ls -l } das Resultat Ihrer
  Aktion.}\\
  \begin{minted}{bash}
  bs@linux-8b19:~/Bs_Prak> ls -l
  insgesamt 16
  lrwxrwxrwx 1 bs users    9 17. Okt 03:36 ltext2.txt -> text2.txt
  lrwxrwxrwx 1 bs users    9 17. Okt 03:37 ltext3.txt -> text3.txt
  -rw-r--r-- 1 bs users 3976 17. Okt 03:30 my_listing.txt
  -rw-r--r-- 1 bs users   28 17. Okt 03:33 text04.txt
  -rw-r--r-- 1 bs users   28 17. Okt 03:33 text2.txt
  -rw-r--r-- 1 bs users   28 17. Okt 03:33 text3.txt
  \end{minted}

  \item \textbf{Editieren Sie \texttt{ltext3.txt}: Wie verändert sich
  \texttt{text3.txt}?}\\
  Die änderung ist von \texttt{ltext3.txt} ist auch in \texttt{text3.txt}
   gespeichert worden.


  \item \textbf{Was passiert, wenn Sie \texttt{ltext2.txt} löschen?}\\
  \texttt{ltext2.txt} wird unbrauchbar. Der symbolische Link funtioniert wie
  ein aus Windows gewöhnliche Verknüpfung: Er verzweigt nur auf die
  Originaldatei und enthält sonst keine Information. Da die Originaldatei nicht
  mehr existiert, funtioniert \texttt{ltext2.txt} auch nicht mehr.

  \item \textbf{Was passiert, wenn Sie \texttt{text3.txt} löschen?}\\
  Beim Erstellen von \texttt{ltext3.txt} wurde eine synchronisierte Kopie
  vom Original \texttt{text3.txt} erstellt. Beim löschen vom Original bleibt
  der Inhalt also im Link erhalten.

  \item \textbf{Demonstrieren Sie die Platzhalterzeichen mit eigenen
  Beispielen.}\\
  \begin{minted}{bash}
  bs@linux-8b19:~/Bs_Prak> ls
  ltext3.txt my_listing.txt text04.txt text2.txt
  text04.txt text2.txt
  bs@linux-8b19:~/Bs_Prak> ls *ext?.txt
  ltext3.txt text2.txt
  bs@linux-8b19:~/Bs_Prak> ls [ml]*
  ltext3.txt my_listing.txt
  bs@linux-8b19:~/Bs_Prak> ls [k-n]*
  ltext3.txt my_listing.txt
  \end{minted}


  \item \textbf{Finden Sie heraus, was die Wirkung der Zeichen \texttt{\$},
  \texttt{\^{}} und \texttt{\textbackslash <} ist. Warum musste der
  Suchausdruck im letzten Beispiel in Anführungszeichen (\emph{quotes})
  gesetzt werden?}\\
  \begin{minted}{bash}
  $ ls -l /etc/ | grep "\<fs"
  # Es werden alle Zeilen ausgegeben, in denen Namen/Wörter vorkommen,
  # die mit "fs" beginnen.

  $ ls -1 /etc/ | grep ^fs
  # Es werden alle Zeilen ausgegeben,  mit "fs" beginnen.

  $ ls -l /etc/ | grep fs$
  # Es werden alle Zeilen ausgegeben, die mit fs enden.
  \end{minted}

  \item \textbf{Geben Sie alle Prozesse aus, deren Kommandozeile mit k
  beginnt.}\\
  \begin{minted}{bash}
  ps aux | grep ^k
  # oder wenn man de Befehle will
  # und nicht einen usernamen hat der mit k anfaengt
  ps aux | grep "\<fs"
  \end{minted}
\end{enumerate}

\section{Vorgänge automatisieren: Shellskripte}
\begin{enumerate}
  \item \textbf{Was tut das oben angegebene Shellskript?}
  Es fragt nach den Anwender an seinem Namen und und gibt dann in einem
  \texttt{here file} eine schmucke ASCII-Begrüßung auf dem Terminal aus.
  Zudem kann man mit dem Flag \texttt{-h} oder \texttt{--help} eine
  Kurzanleitung sich ausgeben lassen. Alle weiteren Argumente werden nicht
  geduldet.

  \item \textbf{Wie bekommen Sie heraus, welche Version des C-Compilers
  \texttt{gcc} auf Ihrer virtuellen Maschine installiert ist?}\\
  \texttt{gcc} ist ein ganz besonders gesittets Programm:
  \begin{minted}{bash}
  $ gcc --version
  \end{minted}

  \item \textbf{Schreiben Sie ein Shellskript \texttt{filecount.sh}, das die
  Anzahl von Dateien der als Option spezifizierten Typen ausgibt. Mit der
  Option \texttt{-h} oder \texttt{--help} soll es diesen Hilfetext ausgeben und
  damit gleichzeitig erklären, was genau Sie implementieren sollen.}
  \\
  \begin{minted}{bash}
cat<<EOF
  usage :
    $0 [ OPTIONS ] DIRECTORY

      Print the number of files in DIRECTORY
      of types given in OPTIONS

      Default: Print numbers of regular files
      in current directory

  OPTIONS :
  FILE type options:
    -f --regular-file  regular files
    -l --symlink       symbolic links
    -D --device        character and block devices
    -d --directory     directories
    -a --all           Include hidden files

  Other options:
    -h --help          Display this text
    -e --echo          Print selected files
    -v --verbose       print debugging messages
    -V --version       print Versionsnumber
EOF
  \end{minted}
Grundidee:
\begin{minted}{bash}
ls -l | egrep ^$TYPE | wc -l
# for echo
ls -l | egrep ^$TYPE | tee
\end{minted}

\end{enumerate}

\subsection{lfilecound.sh}
Die Grundidee wurde uns ja schon gegeben. Beim Aufruf von \texttt{ls -l} ist
der erste Character ein Bezeichner \texttt{\$TYPE} für den \texttt{file type}.
Diese Zeilen ziehen ziehen wir mit \texttt{egrep \^\$TYPE} heraus und lassen
anschließend die Zeilen zählen. \\

Die Optionsflag und das zu untersuchene Verzeichnis werden in einer
über die Argumente laufenden while-Schleife mit eingebetterter
Fall(case)-Unterscheidung herausgefilter und bearbeitet (\texttt{ mit \$1 und
shift}).\\

Die Abarbeitung der zusammengesetzten Flags stellt ist etwas kniffelig und
mühsam (da entweder viele Fallunterscheidungen implementiert werden müssten
oder es einen geheimen Trick gibt,den keiner verraten möchte.)
und wird eventuell nicht mehr implementiert. \\

Der verbose modus wird mit in if-Blöcken eingebetteten
\texttt{echo}s verwirklicht.

Wir haben uns entschlossen, dass die type-flags sich nciht gegenseitig
ausschließen und man so eine Komposition an unteschiedlichsten Typen zählen
lassen kann.

Für den all-mode wird das ls optionsflag mit \texttt{A  (also ls -lA ) }
erweitert, da dies die Option almost-all ist, welche die hidden files darstellt,
jedoch das current und parent directory nicht mitauflistet. \texttt{(. ..)}


Der folgende Code der \texttt{filecount.sh} ist noch nicht lauffähig und dient
nur der Übersicht für den morgigen Tag.
\inputminted{bash}{lfilecound.sh}

\section{Quellcode}

Der komplette Quellcode zu den Aufgaben sowie zu diesem Dokument kann unter
folgenden URL das Reposetory mittels Git geklont werden:\\
\url{https://bitbucket.org/bibutNkafawi/a01_der_umgang_mit_linux}
\\

\end{document}
