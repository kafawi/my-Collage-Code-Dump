% PROTOKOLL BSP 02
\documentclass[
   draft=false
  ,paper=a4
  ,twoside=false
  ,fontsize=11pt
  ,headsepline
  ,BCOR10mm
  ,DIV11
  ,parskip=full+
]{scrartcl} % copied from Thesis Template from HAW

\usepackage[ngerman,english]{babel}
\usepackage[T1]{fontenc}
\usepackage[utf8]{inputenc}

\usepackage[german,refpage]{nomencl}

\usepackage{float}
\usepackage{enumitem}
\usepackage{hyperref} % for a better experience

\hypersetup{
   colorlinks=true % if false - links get colored frames
  ,linkcolor=black % color of tex intern links
  ,urlcolor=blue   % color of url links
}

\usepackage{amsmath}

\usepackage{array}   % for \newcolumntype macro
\newcolumntype{L}{>{$}l<{$}} % math-mode version of "l" column type
\newcolumntype{R}{>{$}r<{$}} % math-mode version of "r" column type
\newcolumntype{C}{>{$}c<{$}} % math-mode version of "c" column type

%\usepackage{listing}
\usepackage{caption}
%\usepackage{xcolor}
%\definecolor{bg}{rgb}{0.60,0.95,0.95}
%using minted because of the hashtag in bash
\usepackage{minted}
% c listing
\newminted{c}{fontsize=\small
             ,fontfamily=tt
             ,linenos
             ,frame=single
             } % \begin{ccode} ... \end{ccode}
\newmintedfile{c}{fontsize=\small
                 ,fontfamily=tt
                 ,linenos
                 ,frame=single
                 ,autogobble
                 } % \cfile{}
% Makefile listing
\newminted{make}{fontsize=\small
             ,fontfamily=tt
             ,linenos
             ,frame=single
             }  % \begin{makecode} ... \end{makecode}
\newmintedfile{make}{fontsize=\small
                    ,fontfamily=tt
                    ,linenos
                    ,frame=single
                    ,autogobble
                    } % \makefile{}
\sloppy
\clubpenalty=10000
\widowpenalty=10000
\displaywidowpenalty=10000

\begin{document}

\selectlanguage{ngerman}
\usemintedstyle{emacs}
% ----------------------------------------------------------------------------
% Titel (erst nach \begin{document}, damit babel bereits voll aktiv ist:
\titlehead{Betriebssysteme WS 2016 Praktikum 03}% optional
\subject{BS Praktikumsaufgabe 03}
\title{Virtuelle Speicherverwaltung}
\subtitle{Version 0.1 - Abgabe am 28. November 2016 16:00}
\author{Alexander Mendel \\ Karl-Fabian Witte}
\date{erstellt am \today}% sinnvoll
%\publishers{Platz für Betreuer o.\,ä.}% optional
% ----------------------------------------------------------------------------
\maketitle% verwendet die zuvor gemachte Angaben zur Gestaltung eines Titels
\begin{abstract}
    Der Mechanismus des virtuellen Pagings wird nachgebildet und
    mit den Algorithmen \textbf{FIFO}, \textbf{CLOCK} und \textbf{LRU} 
    getestet. Dabei ist die Datei 
    \emph{pagefile.bin} der Festplattenersatz. Der physikalische Speicher 
    wird als
    "{}Shared Memory"{} abgebildet. Anstelle von Interrupts werden
    Signale verwendet. Ein Quellcodegerüst wurde uns vorab zur 
    Verfügung gestellt, welches einige Objekte im Vorhinein definiert. 
    \end{abstract}
\tableofcontents
% ----------------------------------------------------------------------------
\flushleft
\section{Entwurf}
Im Großen und Ganzen werden die Funktionsköpfe und Definitionen, welche 
uns gegeben wurden, beibehalten und damit das Vorhaben realisiert. Nur kleine
Änderungen haben wir uns erlaubt. 

    \subsection{Grundgerüst}
        Die Simulation besteht aus 2 Prozessen, die sich einen gemeinsamen 
        Speicherbereich teilen. Es wird zunächst \texttt{mmanage} aufgerufen. 
        Dieser Prozess legt den gemeinsamen Speicher fest und verwaltet. Zudem 
        legt er auch den SWAP bzw. Auslagerung auf der Festplatte an und 
        verwaltet diesen als einziger. Nachdem \texttt{mmanage} alles 
        initialisiert hat, wartet er auf ein Signal, damit er unter anderem 
        die page fault routine \texttt{allocate\_page()} ausführen kann, worin 
        er den nächsten Pageslot aussucht, ggf. die Daten der alten page ins 
        \texttt{pagefile.bin} schreibt, den frame der benötigten page holt 
        und die daten als page in den gemeinsamen Speicherbereich schreibt. Danach
        wird der \texttt{vmappl} wieder frei gegeben.
        
        \texttt{vmappl} sortiert Daten und interagiert mit dem gemeinsamen 
        Speicher, der viel zu klein ist, um alle Daten zu halten. Wenn die 
        gewünschte Seite zum Lesen bzw. Schreiben nicht im gemeinsamen 
        Speicher liegt, haben wir einen \textit{page fault} und ein Signal 
        wird an \texttt{mmanage} geschickt und \texttt{vmappl} blockiert sich
        selbst.

    \subsection{Schlüsselobjekte}

        \subsubsection{Datei statt Festplatte}
        Anstelle des gesonderten Festplattenbereiches, auf der die 
        auszulagernden Daten gespeichert werden, erstellen wir mit \texttt{fopen()}
        eine Dateien, aus der wir bei einem page fault die gewünschte Daten 
        in den RAM-Ersatz lesen und ggf. (dirty) Daten vom RAM Ersatz in die 
        Datei schreiben. Dabei wird die Datei mit sehr großen Zahlen initalisiert
        $( 0 \leq x \leq RAND_MAX = 2^{32}-1 \widehat{=} \mathcal{O}(10^{10}))$.
        Da im Anwendungsprogramm \texttt{vmappl} nur Zahlen erzeugt werden, 
        die kleiner sind als 1000, kann man so erkennen, bis wohin unser 
        \texttt{mmanage} die neuen Daten auslagert. Wir verwenden ein festes 
        Format, mit der wir die Daten in die Datei schreiben ("{}\%10d"{} + 
        Delimiter). Zwei Funktionen in \texttt{mmanage} greifen auf diese
        Datei direkt zu: 
        \begin{itemize}
                \item[\texttt{fetch\_page()}:] greift lesend auf die Datei zu,
                indem der Positionszeiger der Datei mit \texttt{fseek()} 
                gesetzt wird und mit \texttt{fscanf()} von dieser Position an 
                gelesen wird.
                \item[\texttt{store\_page()}:] greift schreibend auf die Datei zu,
                indem der Positionszeiger der Datei mit \texttt{fseek()} 
                gesetzt wird und mit \texttt{fprintf()} von dieser Position an
                beschrieben wird.
        \end{itemize}

        \subsubsection{Shared Memory statt RAM}
        Der Arbeitsspeicher wird mit einem Prozess übergreifenden 
        Speicherbereich (Shared Memory) simuliert.
        Wir verwenden die ältere System V Form. Dabei gehen wir wie folgt vor.
        \texttt{vmem\_init} erstellt den gemeinsamen Speicher mit 
        \texttt{shmget()}, wobei IPC\_CREATE mit als Argument übergeben wird,
        mit dem der Speicherbereich alloziert wird. Mit der erhaltenden ID 
        erhält man durch \texttt{shmat()} einen Zeiger auf den Bereich, welcher
        zu der Struktur aus \texttt{vmem.h} gecastet wird. 
        In \texttt{vmmacces.c : vm\_init()} funktioniert das ähnlich , jedoch 
        wird hier der IPC\_CREATE nicht übergeben. Damit beide Prozesse den 
        selben Speicherbereich erhalten, wird \texttt{shmget()} der selbe 
        key SHMNAME übergeben.
        Die Zerstörung des Shared Memory erfolgt in \texttt{clean\_up()} in 
        \texttt{mmanage}. Hier wird \texttt{shmdt} zum Lösen und danach 
        \texttt{shmctl()} mit IPC\_RMID  zum Zerstören aufgerufen.
        
        Der Inhalt der \texttt{vmem} Structur wird in 
        \texttt{mmanage: vemm\_init()} gefüllt. Es wird zuerst alles auf 
        default Werte bzw. VOID\_IDX gesetzt. Auch die Prozess ID von 
        \texttt{mmanage} wird in dieser gespeichert, damit \texttt{vmappl} bei 
        einem Seitenfehler ein Signal an \texttt{mmanage} schicken kann.
        Der Semaphor zum blocken von \texttt{vmappl} wird ebenfalls dort 
        erstellt.

        \subsubsection{Signal statt Interrupts}
        Ein Betriebssystem operiert bei \textit{page faults} mit Interrupts. Wir
        operieren hingegen mit Signalen. Bei einem \textit{page fault} schickt
        der \texttt{vmappl} Prozess über \texttt{vmaccess} Funktionen bei einem
        \textit{page fault} (PRESENT\_FLAG ist nicht gesetzt -> Seite ist nicht
        im Speicher (vmem)) ein Signal (SIGUSR1) an den \texttt{mmanage} Prozess und 
        blockiert sich selber.  
        Dieser reagiert mit \texttt{sighandler()} und ruft die Funktion
        \texttt{allocate\_page()} auf, welches die geforderte Seite in den 
        Speicher lädt.

        Mit dem Signal SIGINT wird \texttt{mmanage} nach dem Zerstören des
        \textit{Shared Memory} durch \texttt{clean\_up()} das Programm beendet.
        Ein weiteres Signal (SIGUSR2) ruft die Funktion \texttt{dump\_pt()}
        auf, welches die Seitentabelle einmal auflistet. Das nutzt uns für das
        Debugging. 
        
        \subsubsection{\texttt{vmem\_read} und \texttt{vmem\_write}}
        Zuerst wird überprüft, ob der Speicher schon bekannt ist und ggf.  bekannt gemacht. \texttt{vmappl} ruft nämlich nicht direkt
        die Funktion \texttt{vm\_init} auf.
        \texttt{vmappl} nimmt aus der quasie Bibliothek \texttt{vmaccess} die 
        Funktionen \texttt{vmem\_read(address)} und \texttt{vmem\_write(address)}.
        Da \texttt{address} eine volle virtuelle Adresse ist, muss ein 
        Seitenindex aus dieser errechnet werden. Dies erfolgt mit der 
        Ganzzahldivision mit der Seitengröße des Speichers VMEM\_PAGESIZE.
        Der Offset wird ähnlich wie der des Seitenindexes berechnet, jedoch mit
        Modulo statt Division.

        Es wird geprüft, ob die Seite schon im Speicher liegt, wenn nicht wird 
        ein Signal gesendet und blockiert (page fault). Nun ist die Seite zum
        Lesen bzw. Schreiben verfügbar. Beim Schreiben wird das DIRTY\_FLAG 
        gesetzt. Beim Lesen und Schreiben wird das USED\_FLAG gesetzt (und für
        LRU wichtig, der jeweilige counter auf 0 gesetzt.)

        Nachdem die Seite nun geladen ist, wird noch der Offset an den 
        Seitenindex angehängt und zurückgegeben. 
        
    \subsection{Algorithmen}
    Welche Seite ersetzt werden soll, entscheidet der Seitenersetzungsalgorithmus.
    Die Adresse der jeweiligen Seitenersetzungsfunktion wird bei der Algorithmusabfrage mit getopt einem Funktionszeiger übergeben.
        \subsubsection{FIFO}
        Der Fifo ist der einfachste und kann mit einer Zeile Code realisiert werden.
        Dabei wird einfach der \texttt{next\_alloc\_idx} um einen erhöht 
        und das Modulo mit der Rahmenanzahl VMEM\_NFRAMES gebildet. Wie ein Ringbuffer.

        \subsubsection{CLOCK}
        Dieser ist dem Fifo relativ ähnlich. Dabei wird ein Zeiger (eigentlich 
        ein Index) wie der Fifo berechnet, jedoch wird mit der page, welche 
        daraus resultiert, das USED\_FLAG geprüft. Wenn es gesetzt ist, wird es
        gelöscht, solange bis der Zeiger auf eine Seite zeigt, die ein 
        gelöschtes USED\_FLAG besitzt und somit raus geworfen wird.

        \subsubsection{LRU}
        Als pseudo Zeit wird ein page counter verwendet, der in jeder page 
        erhöht wird, wenn ein Schreibe- oder Lesezugriff erfolgt. 
        Der Zähler der benutzten Seite wird auf Null gesetzt. 
        Der Algorithmus selber sucht alle Seiten einmal ab und speichert die 
        Seite mit dem größten Zählerstand und gibt diesen zurück.


    \subsection{Sonstiges}
        Die Auswahl der Algorithmen wird mittels der getopt Funktion von GNU 
        realisiert. Wie das genau aussehen wird, ist noch nicht entschieden.
        Alle missglückten Systemaufrufe führen zum Absturz.
\section{How to compile - Makefile}
     Da wir zwei Prograimme übersetzten werden, werden die Linkeranweisungen 
     aufgeteilt. Mit dem Befehl make all bzw. run\_all werden alle Programme 
     erstellt bzw. ausgeführt.
%\section{Quellcode}

\end{document}
