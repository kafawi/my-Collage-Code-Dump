% PROTOKOLL Action Item
\documentclass[
   draft=false
  ,paper=a4
  ,twoside=false
  ,fontsize=11pt
  ,headsepline
  ,DIV11
  ,parskip=full+
]{scrartcl} % copied from Thesis Template from HAW

\usepackage[ngerman,english]{babel}
\usepackage[T1]{fontenc}
\usepackage[utf8]{inputenc}

\usepackage[
    left  =4em
   ,right =4em
   ,top   =5em
   ,bottom=5em
]{geometry}

\usepackage{longtable}
\usepackage[german,refpage]{nomencl}

\usepackage{float}
\usepackage{enumitem}
\usepackage{hyperref} % for a better experience

\hypersetup{
   colorlinks=true % if false - links get colored frames
  ,linkcolor=black % color of tex intern links
  ,urlcolor=blue   % color of url links
}

\usepackage{amsmath}

\usepackage{array}   % for \newcolumntype macro
\newcolumntype{L}{>{$}l<{$}} % math-mode version of "l" column type
\newcolumntype{R}{>{$}r<{$}} % math-mode version of "r" column type
\newcolumntype{C}{>{$}c<{$}} % math-mode version of "c" column type

\usepackage{listing}
\usepackage{caption}
\usepackage{colortbl}
\definecolor{tabgrey}{rgb}{0.85,0.85,0.85}
%using minted because of the hashtag in bash

\sloppy
\clubpenalty=10000
\widowpenalty=10000
\displaywidowpenalty=10000

\begin{document}

\selectlanguage{ngerman}
% ----------------------------------------------------------------------------
% ---------------------------------------------------------- HIER WAS MACHEN -
% -------------------------------- Metadaten wie namen und Gruppentreffen etc-

\def\titel{Requirements - Leuchten am 10.04.2017}

\def\myName{Karl-Fabian Witte}
\def\myEmail{Karl-Fabian.Witte@haw-hamburg.de}


\def\teilnehmer{ 
	& Lennart Hartmann & Eduard Veit \\
	& Alexander Mendel   & Nils Eggebrecht\\
    & Karl-Fabian Witte   & \\
}

\def\zurKenntnis {
	& T.Lehmann(HAW Hamburg) & Profs(HAW Hamburg) \\
}
% -------------------------------------------------- HIER AUFHÖREN ----------



% ------------------------------------------ einige strukturell Definitionen
\newlength{\txtw} %definiere neue länge
\setlength{\txtw}{\textwidth} %setze neue länge auf textbreite
\addtolength{\txtw}{-10\tabcolsep} %subtrahiere -8\cdot textbreite von asdf

\def\me{\myName \newline \footnotesize{\url{\myEmail} } }

\def\tablehead{
	\hline 
	\rowcolor{tabgrey}
	\textbf{Typ} & 
	\textbf{Resultat regulär} & 
	\textbf{gewünschte Rutsche voll} \\
	\hline 
	\endhead}

% ------------------------------------------------------------------ Inhalt	
\begin{tabular}{p{0.65\txtw} p{0.35\txtw}}
	\textbf{\Large{\titel}} & \me  \\
\end{tabular}

\begin{tabular}{l p{0.4\txtw} p{0.4\txtw} }
	Teilnehmer: \teilnehmer
	Zur Kenntnis: \zurKenntnis
\end{tabular}
\texttt {Dieses Dokument ist vorerst Abgeschlossen. Es werden keine Änderungen benötigt.}


% vim: set spell spelllang=de :EOF
%%%%%%%%%%%%%%%%%%%%%%%%%%%%%%%%%%%%%%%%%%%%%%%

\def\tablehead{
	\hline 
	\rowcolor{tabgrey}
	\textbf{Typ} & 
	\textbf{Grund des Leuchtens}  
 \\
	\hline 
	\endhead}

\subsection*{\titel} 
\vspace{-1em}
\begin{longtable}{
	|p{0.45\txtw} % Nr.
	|p{0.45\txtw} % Art
|}
	\tablehead
 
	gelbe Leuchte blinkt & Flache Werkstücke auf Band1 erkannt. (wird aussortiert)
	\\ \hline 

	grün leuchtend & Bandanlage in Betrieb  
	\\ \hline

	rot 1Hz  & anstehend unquittiert  \\ \hline

	Brot 0,5Hz & gegangen quittiert
	\\ \hline		

    rot leuchtend & anstehend quittiert  
	\\ \hline
\end{longtable}

\vspace{-2.5em}
\footnotesize

\normalsize

\flushleft

\end{document}
% vim: set spell spelllang=de :EOF

%%%%%%%%%%%%%%%%%%%%%%%%%%%%%%%%%%%%%%%%%%%%%%%
