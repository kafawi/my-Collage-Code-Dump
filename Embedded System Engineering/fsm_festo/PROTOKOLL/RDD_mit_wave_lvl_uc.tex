% RDD - entsprechend der Vorlage von Thomas Lehmann
\documentclass[
   draft=false
  ,paper=a4
  ,twoside=true
  ,fontsize=11pt
  ,headsepline
  ,DIV11
  ,parskip=full+
]{scrartcl} % copied from Thesis Template from HAW


%--------------------------------------- HIER Versionsnummer inkrementieren, 
%------------------------------- auch gerne 0.1.1 ... wenn nur minor changes
\def\ver{0.2}


\usepackage{scrpage2} %Sorgt für erweiterte Formatierungsmöglichkeiten bei KOMA, wie eben Seitenzahlen-Position
\usepackage[ngerman,english]{babel}
\usepackage[T1]{fontenc}
\usepackage[utf8]{inputenc}
\addtokomafont{disposition}{\rmfamily}
\renewcommand{\rmdefault}{ptm}
\renewcommand*{\familydefault}{\rmdefault}

%== Formatierungsschemata ==%
\pagestyle{scrheadings} %Seitenstil festlegen, damit die folgenden Einträge auch wirksam sind
\cfoot{} %center, Fuß, {} = ohne Eintrag
\chead{} %center, Kopf, {} = ohne Eintrag
\ofoot{\pagemark} %Außen, Fuß, Seitenzahl
\ohead{\headmark}

\renewcommand*\titlepagestyle{empty} % unterdrückt seitenzahl
\usepackage[
    left  =7em
   ,right =7em
   ,top   =8em
   ,bottom=8em
]{geometry}


\usepackage{longtable}
\usepackage[german,refpage]{nomencl}

\usepackage{float}
\usepackage{enumitem}
\usepackage{hyperref} % for a better experience
\urlstyle{same}

\hypersetup{
   colorlinks=true % if false - links get colored frames
  ,linkcolor=black % color of tex intern links
  ,urlcolor=blue   % color of url links
}

\usepackage{amsmath}
\usepackage{graphicx}


\usepackage{array}   % for \newcolumntype macro
\newcolumntype{L}{>{$}l<{$}} % math-mode version of "l" column type
\newcolumntype{R}{>{$}r<{$}} % math-mode version of "r" column type
\newcolumntype{C}{>{$}c<{$}} % math-mode version of "c" column type

\usepackage{listing}
\usepackage{caption}
\usepackage{colortbl}
\definecolor{tabgrey}{rgb}{0.85,0.85,0.85}

\sloppy
\clubpenalty=10000
\widowpenalty=10000
\displaywidowpenalty=10000

\begin{document}
\selectlanguage{ngerman}
\begin{titlepage}
\title{Usecase Tabellen \\ (RDD)}
%\subject{subject: Beispiel}
\subtitle{Version \ver \\ \vspace{1em} ESEP – Praktikum – Sommersemester 2017 }
\author{ LANKE \\
\normalsize{
\begin{tabular}{l l l l} 
Hartmann   & Lennart     & 1234567 & \url{Lennart.Hartmann@haw-hamburg.de} \\
Mendel     & Alexander   & 2188808 & \url{Alexander.Mendel@haw-hamburg.de} \\
Eggebrecht & Nils        & 1234567 & \url{Nils.Eggebrecht@haw-hamburg.de} \\
Witte      & Karl-Fabian & 2246435 & \url{Karl-Fabian.Witte@haw-hamburg.de} \\
Veit       & Eduard      & 1234567 &  \url{Eduard.Veit@haw-hamburg.de} \\
\end{tabular}}}
\dedication{
\normalsize{
\begin{tabular}{|l |l |l |p{25em}|}
\multicolumn{4}{l}{ \Large{Änderungshistorie:}} \\
\hline 
\rowcolor{tabgrey} \textbf{Version} & \textbf{Autor} & \textbf{Datum} & \textbf{Anmerkungen/Änderungen} \\
\hline
0.1 &
K.Witte &
04.04.2017 &
Aus der Vorlage (Version 0.5 ) von Prof Lehmann doc2tex, \newline um es in Git besser pflegen zu können \\
\hline
0.1.1 &
K.Witte &
11.04.2017 &
Es wurden Tabellenvorlagen für die Requirements und Use Cases hinzugefügt (Wave und Kite lvl)\\ 
\hline
0.2 & 
A. Mendel &
18.04.2017 &
Getrennte Version des RDDs für die Tabellen\\
\hline

\end{tabular}
}}
\date{Hamburg, den \today }

\maketitle
%\texttt {Dieses Dokument ist noch nicht Abgeschlossen. Es werden noch Änderungen benötigt.}
\end{titlepage}

\newpage
 

\setcounter{page}{1}
\flushleft

\pagenumbering{roman}

\newpage
\tableofcontents
\newpage



\pagenumbering {arabic}
\section{Wave Level}
!!Namensgebung der Akteure vorest entsprechend Meeting 24.03.17 (Deutsche Namen)
\newline
\newline
\begin{table}[htp]
\caption{WAVE LVL USE CASE}
\label{tab:usecase_wave}

\begin{tabular}{|p{0.3\linewidth}| p{0.7\linewidth} |}
	\hline 
	\rowcolor{tabgrey} \textbf{Name} & \textbf{Puck mit Bohrung unten regulär} \\
	\hline

	Akteur & 
	\frqq Lichtschranke.inHoehenmessung\flqq, \frqq Lichtschranke.einlauf\flqq, \frqq Messung.hoehe\flqq und alle sonstigen eigentlich auch \\ \hline
	Auslösendes Ereignis &
		Das Ergebnis von \frqq Messung.hoehe\flqq  \\ \hline
	Kurzbeschreibung & 
		Die Hoehenmessung ergibt einen Wert, der nicht einer Bohrung, nicht dem eines flachen Werkstückes oder nicht eines Bohrungstyps (1, 2, 4, 5) entspricht und sortiert dementsprechend auf die Rutsche aus \\ \hline
		
	Vorbedingungen & 
		\frqq Lichtschranke.rutscheVoll\flqq frei\\ \hline
	Essentielle Schritte &
		\begin{tabular}{|p{0.4\linewidth}|p{0.52\linewidth}|}
		\hline
			\rowcolor{tabgrey} \textbf{Intention \newline der Systemumgebung} & \textbf{Reaktion des Systems} \\ \hline \rowcolor{white}

			Schritt 1: \frqq Messung.hoehe\flqq ergibt \emph{Bohrung unten regulär} &
				Reaktion 1: Typerkennungsausgabe auf Konsole \newline
				
		\end{tabular} \\ \hline
	
	Ausnahmefälle &
		Später einfügen, Initial nur Normalverhalten \\ \hline
	Nachbedingungen & 
		Können wir noch nicht festlegen\\ \hline
	Zeitverhalten &
		...(muss hier etwas hin, wegen wenn zu früh oder zu spät?)  \\ \hline
	Verfügbarkeit & 
		...(So etwas wie erwartete / notwendige MTBF o.ä.) \\ \hline
	Fragen/Kommentare &
		Siehe \emph {Ausnahmefälle, Zeitverhalten, Verfügbarkeit} - das müssen wir noch besprechen \\ \hline
\end{tabular}
\end{table}
\\
\begin{table}[htp]
\caption{WAVE LVL USE CASE}
\label{tab:usecase_wave}
\begin{tabular}{|p{0.3\linewidth}| p{0.7\linewidth} |}
	\hline 
	\rowcolor{tabgrey} \textbf{Name} & \textbf{Puck in richtiger Reihenfolge regulär} \\
	\hline

	Akteur & 
	\frqq Lichtschranke.inHoehenmessung\flqq, \frqq Lichtschranke.einlauf\flqq, \frqq Lichtschranke.inWeiche\flqq, \frqq weiche.oeffnen\flqq, \frqq Messung.hoehe\flqq, \frqq Messung.metall\flqq und alle sonstigen eigentlich auch \\ \hline
	Auslösendes Ereignis &
		Das Ergebnis von \frqq Messung.hoehe\flqq und \frqq Messung.metall\flqq \\ \hline
	Kurzbeschreibung & 
		Die Höhenmessung bzw. Metallerkennung ergibt einen Wert, der entsprechend der Reihenfolgeerkennung stimmt (Genauer definieren) \\ \hline
		
	Vorbedingungen & 
		Zustand: \frqq OK\flqq \\ \hline
	Essentielle Schritte &
		\begin{tabular}{|p{0.4\linewidth}|p{0.52\linewidth}|}
		\hline
			\rowcolor{tabgrey} \textbf{Intention \newline der Systemumgebung} & \textbf{Reaktion des Systems} \\ \hline \rowcolor{white}

			Schritt 1: \frqq Messung.hoehe \flqq und \frqq Messung.metall\flqq ergibt entsprechend der Reihenfolge \emph{richtige Reihenfolge regulär} &
				Reaktion 1: Typerkennungsausgabe auf Konsole \newline \\ \hline & Reaktion 2: \frqq Weiche.oeffnen\flqq wenn \frqq Lichtschranke.inWeiche\flqq
				
		\end{tabular} \\ \hline
	
	Ausnahmefälle &
		Später einfügen, Initial nur Normalverhalten \\ \hline
	Nachbedingungen & 
	 	...(\frqq Lichtschranke.auslauf\flqq wird nach Zeit \emph{xz} durch den Puk aktiviert)\\ \hline
	Zeitverhalten &
		...(muss hier etwas hin, wegen wenn zu früh oder zu spät?)  \\ \hline
	Verfügbarkeit & 
		...(So etwas wie erwartete / notwendige MTBF o.ä.) \\ \hline
	Fragen/Kommentare &
		Siehe \emph {Ausnahmefälle, Zeitverhalten, Verfügbarkeit} - das müssen wir noch besprechen \\ \hline
\end{tabular}
\newline
\newline
\end{table}

\begin{table}[htp]
\caption{WAVE LVL USE CASE}
\label{tab:usecase_wave}
\begin{tabular}{|p{0.3\linewidth}| p{0.7\linewidth} |}
	\hline 
	\rowcolor{tabgrey} \textbf{Name} & \textbf{Puck in falscher Reihenfolge regulär} \\
	\hline

	Akteur & 
	\frqq Lichtschranke.inHoehenmessung\flqq, \frqq Lichtschranke.einlauf\flqq, \frqq Messung.hoehe\flqq, \frqq Messung.metall\flqq und alle sonstigen eigentlich auch \\ \hline
	Auslösendes Ereignis &
		Das Ergebnis von \frqq Messung.hoehe\flqq und \frqq Messung.metall\flqq \\ \hline
	Kurzbeschreibung & 
		Die Hoehenmessung bzw. Metallerkennung ergibt einen Wert, der entsprechend der Reihenfolgeerkennung \emph{nicht} stimmt \\ \hline
		
	Vorbedingungen & 
		\frqq Lichtschranke.rutscheVoll\flqq frei \\ \hline
	Essentielle Schritte &
		\begin{tabular}{|p{0.4\linewidth}|p{0.52\linewidth}|}
		\hline
			\rowcolor{tabgrey} \textbf{Intention \newline der Systemumgebung} & \textbf{Reaktion des Systems} \\ \hline \rowcolor{white}

			Schritt 1: \frqq Messung.hoehe \flqq und \frqq Messung.metall\flqq ergibt entsprechend der Reihenfolge \emph{falsche Reihenfolge regulär} &
				Reaktion 1: Typerkennungsausgabe auf Konsole \newline \\
				\hline & Reaktion 2: \frqq Weiche.oeffnen\flqq auf Band2, nachdem der Puk \frqq Lichtschranke.inWeiche\flqq auf Band2 erreicht hat
				
		\end{tabular} \\ \hline
	
	Ausnahmefälle &
		Später einfügen, Initial nur Normalverhalten \\ \hline
	Nachbedingungen & 
	 	nüscht\\ \hline
	Zeitverhalten &
		...(muss hier etwas hin, wegen wenn zu früh oder zu spät?)  \\ \hline
	Verfügbarkeit & 
		...(So etwas wie erwartete / notwendige MTBF o.ä.) \\ \hline
	Fragen/Kommentare &
		Siehe \emph {Ausnahmefälle, Zeitverhalten, Verfügbarkeit} - das müssen wir noch besprechen \\ \hline
\end{tabular}
\newline
\newline
\end{table}

\begin{table}[htp]
\caption{WAVE LVL USE CASE}
\label{tab:usecase_wave}
\begin{tabular}{|p{0.3\linewidth}| p{0.7\linewidth} |}
	\hline 
	\rowcolor{tabgrey} \textbf{Name} & \textbf{flacher Puk regulär} \\
	\hline

	Akteur & 
	\frqq Lichtschranke.inHoehenmessung\flqq, \frqq Lichtschranke.einlauf\flqq, \frqq Messung.hoehe\flqq, \frqq Messung.metall\flqq und alle sonstigen eigentlich auch \\ \hline
	Auslösendes Ereignis &
		Das Ergebnis von \frqq Messung.hoehe\flqq  \\ \hline
	Kurzbeschreibung & 
		Die Hoehenmessung ergibt einen Wert für einen \emph{flachen Puk} \\ \hline
		
	Vorbedingungen & 
		\frqq Lichtschranke.rutscheVoll\flqq frei UND Puk befindet sich auf nicht auf Band2 \\ \hline
	Essentielle Schritte &
		\begin{tabular}{|p{0.4\linewidth}|p{0.52\linewidth}|}
		\hline
			\rowcolor{tabgrey} \textbf{Intention \newline der Systemumgebung} & \textbf{Reaktion des Systems} \\ \hline \rowcolor{white}

			Schritt 1: \frqq Messung.hoehe \flqq  ergibt \emph{flacher Puk regulär} &
				Reaktion 1: \frqq Ampel.gelb \flqq blinkt \newline \\ \hline
			
				
		\end{tabular} \\ \hline
	
	Ausnahmefälle &
		Später einfügen, Initial nur Normalverhalten \\ \hline
	Nachbedingungen & 
	 	nüscht \\ \hline
	Zeitverhalten &
		...(muss hier etwas hin, wegen wenn zu früh oder zu spät?)  \\ \hline
	Verfügbarkeit & 
		...(So etwas wie erwartete / notwendige MTBF o.ä.) \\ \hline
	Fragen/Kommentare &
		Siehe \emph {Ausnahmefälle, Zeitverhalten, Verfügbarkeit} - das müssen wir noch besprechen \\ \hline
\end{tabular}
\newline
\newline
\end{table}





\begin{table}[htp]
\caption{WAVE LVL USE CASE}
\label{tab:usecase_wave}
\begin{tabular}{|p{0.3\linewidth}| p{0.7\linewidth} |}
	\hline 
	\rowcolor{tabgrey} \textbf{Name} & \textbf{ein Laufband ist leer} \\
	\hline

	Akteur & 
	\frqq Lichtschranke.*\flqq \\ \hline
	Auslösendes Ereignis &
		Lichtschranken von \emph{Band1} oder \emph{Band2} seit Zeit \emph{xy} nicht ausgelöst  \\ \hline
	Kurzbeschreibung & 
		Es ist auf  \emph{Band1} oder \emph{Band2} kein Puks mehr bzw. die Puk-Liste ist leer \\ \hline
		
	Vorbedingungen & 
		Zustand: \frqq OK\flqq \\ \hline
	Essentielle Schritte &
		\begin{tabular}{|p{0.4\linewidth}|p{0.52\linewidth}|}
		\hline
			\rowcolor{tabgrey} \textbf{Intention \newline der Systemumgebung} & \textbf{Reaktion des Systems} \\ \hline \rowcolor{white}

			Schritt 1: \frqq Lichtschranke.* \flqq von  \emph{Band1} oder \emph{Band2} löst kein Ereignis seit Zeit \emph{xy} aus, \newline führt zu \emph{ein Laufband leer} &
				Reaktion 1: \flqq Motor.stopp\frqq wird ausgelöst  \newline \\ \hline
			
				
		\end{tabular} \\ \hline
	
	Ausnahmefälle &
		Später einfügen, Initial nur Normalverhalten \\ \hline
	Nachbedingungen & 
	 	...\\ \hline
	Zeitverhalten &
		...(muss hier etwas hin, wegen wenn zu früh oder zu spät?)  \\ \hline
	Verfügbarkeit & 
		...(So etwas wie erwartete / notwendige MTBF o.ä.) \\ \hline
	Fragen/Kommentare &
		Siehe \emph {Ausnahmefälle, Zeitverhalten, Verfügbarkeit} - das müssen wir noch besprechen \\ \hline
\end{tabular}
\newline
\newline
\end{table}

\begin{table}[htp]
\caption{WAVE LVL USE CASE}
\label{tab:usecase_wave}
\begin{tabular}{|p{0.3\linewidth}| p{0.7\linewidth} |}
	\hline 
	\rowcolor{tabgrey} \textbf{Name} & \textbf{Puk verschwindet} \\
	\hline

	Akteur & 
	\frqq Timer\flqq \\ \hline
	Auslösendes Ereignis &
		Timer für einen Puk läuft ab/wird zu früh unterbrochen  \\ \hline
	Kurzbeschreibung & 
		Es ist auf  \emph{Band1} oder \emph{Band2} nach der Puk-Liste ist noch ein Puk und dieser erreicht die Lichtschranke zur Timerunterbrechung zu früh oder spät \\ \hline
		
	Vorbedingungen & 
		Zustand: \frqq OK\flqq :/odernix? \\ \hline 
	Essentielle Schritte &
		\begin{tabular}{|p{0.4\linewidth}|p{0.52\linewidth}|}
		\hline
			\rowcolor{tabgrey} \textbf{Intention \newline der Systemumgebung} & \textbf{Reaktion des Systems} \\ \hline \rowcolor{white}

			Schritt 1: Timer läuft ab/wird zu früh unterbrochen\newline führt zu \emph{Puk verschwindet} &
				Reaktion 1: \flqq Motor.stopp\frqq wird für  \emph{Band1} und \emph{Band2} ausgelöst  \newline \\ \hline  & 
				Reaktion 2: Fehlermeldung	\\ \hline
			
				
		\end{tabular} \\ \hline
	
	Ausnahmefälle &
		Später einfügen, Initial nur Normalverhalten \\ \hline
	Nachbedingungen & 
	 	...\\ \hline
	Zeitverhalten &
		...(muss hier etwas hin, wegen wenn zu früh oder zu spät?)  \\ \hline
	Verfügbarkeit & 
		...(So etwas wie erwartete / notwendige MTBF o.ä.) \\ \hline
	Fragen/Kommentare &
		Siehe \emph {Ausnahmefälle, Zeitverhalten, Verfügbarkeit} - das müssen wir noch besprechen \\ \hline
\end{tabular}
\newline
\newline
\end{table}


\begin{table}[htp]
\caption{WAVE LVL USE CASE}
\label{tab:usecase_wave}
\begin{tabular}{|p{0.3\linewidth}| p{0.7\linewidth} |}
	\hline 
	\rowcolor{tabgrey} \textbf{Name} & \textbf{Puk unregulär hinzugefügt} \\
	\hline

	Akteur & 
	\frqq Lichtschranke.inHoehenmessung\flqq, \frqq Lichtschranke.inWeiche \flqq, \frqq Lichtschranke.auslauf \flqq \\ \hline
	Auslösendes Ereignis &
		Genannte Lichtschranken von \emph{Band1} oder \emph{Band2} werden ausgelöst  \\ \hline
	Kurzbeschreibung & 
		Die genannten Lichtschranken werden ausgelöst, obwohl kein anderes Ereignis es zu dieser Zeit machen sollte \\ \hline
		
	Vorbedingungen & 
		Zustand: \frqq OK\flqq \\ \hline
	Essentielle Schritte &
		\begin{tabular}{|p{0.4\linewidth}|p{0.52\linewidth}|}
		\hline
			\rowcolor{tabgrey} \textbf{Intention \newline der Systemumgebung} & \textbf{Reaktion des Systems} \\ \hline \rowcolor{white}

			Schritt 1: Die Lichtschranken von  \emph{Band1} oder \emph{Band2} löst ein Ereignis aus, obwohl eigentlich kein Ereignis ansteht, \newline führt zu \emph{Puk unregulär hinzugefügt} &
				Reaktion 1: \flqq Motor.stopp\frqq wird für  \emph{Band1} und \emph{Band2} ausgelöst  \newline \\ \hline & 
				Reaktion 2: Fehlermeldung	\\ \hline
				
		\end{tabular} \\ \hline
	
	Ausnahmefälle &
		Später einfügen, Initial nur Normalverhalten \\ \hline
	Nachbedingungen & 
	 	...\\ \hline
	Zeitverhalten &
		...(muss hier etwas hin, wegen wenn zu früh oder zu spät?)  \\ \hline
	Verfügbarkeit & 
		...(So etwas wie erwartete / notwendige MTBF o.ä.) \\ \hline
	Fragen/Kommentare &
		Siehe \emph {Ausnahmefälle, Zeitverhalten, Verfügbarkeit} - das müssen wir noch besprechen \\ \hline
\end{tabular}
\newline
\newline
\end{table}

\begin{table}[htp]
\caption{WAVE LVL USE CASE}
\label{tab:usecase_wave}
\begin{tabular}{|p{0.3\linewidth}| p{0.7\linewidth} |}
	\hline 
	\rowcolor{tabgrey} \textbf{Name} & \textbf{beide Rutschen sind voll} \\
	\hline

	Akteur & 
	\frqq Lichtschranke.rutscheVoll\flqq beider Bänder \\ \hline
	Auslösendes Ereignis &
		Genannte Lichtschranken von \emph{Band1} oder \emph{Band2} werden ausgelöst  \\ \hline
	Kurzbeschreibung & 
		Die genannten Lichtschranken werden ausgelöst und die Laufbänder stoppen \\ \hline
		
	Vorbedingungen & 
		Zustand: \frqq OK\flqq :/oderwieoderwat?\\ \hline
	Essentielle Schritte &
		\begin{tabular}{|p{0.4\linewidth}|p{0.52\linewidth}|}
		\hline
			\rowcolor{tabgrey} \textbf{Intention \newline der Systemumgebung} & \textbf{Reaktion des Systems} \\ \hline \rowcolor{white}

			Schritt 1: \flqq Lichtschranke.rutscheVoll \frqq von  \emph{Band1} oder \emph{Band2} löst ein Ereignis aus \newline führt zu \emph{beide Rutschen voll} &
				Reaktion 1: \flqq Motor.stopp\frqq wird für  \emph{Band1} und \emph{Band2} ausgelöst  \newline \\ \hline & 
				Reaktion 2: Fehlermeldung			
				
		\end{tabular} \\ \hline
	
	Ausnahmefälle &
		Später einfügen, Initial nur Normalverhalten \\ \hline
	Nachbedingungen & 
	 	...\\ \hline
	Zeitverhalten &
		...(muss hier etwas hin, wegen wenn zu früh oder zu spät?)  \\ \hline
	Verfügbarkeit & 
		...(So etwas wie erwartete / notwendige MTBF o.ä.) \\ \hline
	Fragen/Kommentare &
		Siehe \emph {Ausnahmefälle, Zeitverhalten, Verfügbarkeit} - das müssen wir noch besprechen \\ \hline
\end{tabular}
\newline
\newline
\end{table}

\begin{table}[htp]
\caption{WAVE LVL USE CASE}
\label{tab:usecase_wave}
\begin{tabular}{|p{0.3\linewidth}| p{0.7\linewidth} |}
	\hline 
	\rowcolor{tabgrey} \textbf{Name} & \textbf{Aus-Taste gedrückt wenn das Laufband an ist} \\
	\hline

	Akteur & 
	\frqq UI.Taste.stop\flqq \\ \hline
	Auslösendes Ereignis &
		\frqq UI.Taste.stop\flqq wird durch mechanischen Eingriff ausgelöst \\ \hline
	Kurzbeschreibung & 
		Die Stopp-Taste wird gedrückt, während die Laufbänder an sind \\ \hline
		
	Vorbedingungen & 
		Zustand des Motors \flqq motor.rechtslauf\frqq \\ \hline
	Essentielle Schritte &
		\begin{tabular}{|p{0.4\linewidth}|p{0.52\linewidth}|}
		\hline
			\rowcolor{tabgrey} \textbf{Intention \newline der Systemumgebung} & \textbf{Reaktion des Systems} \\ \hline \rowcolor{white}

			Schritt 1: \frqq UI.Taste.stop\flqq löst ein Ereignis aus \newline führt zu \emph{Aus-Taste gedrückt wenn das Laufband an ist} &
				Reaktion 1: \flqq Motor.stopp\frqq wird für  \emph{Band1} und \emph{Band2} ausgelöst  \newline \\ \hline & 
				Reaktion 2: \frqq UI.LED.stop\flqq leuchtet	\\ \hline &
				Reaktion 3: Zustände werden gespeichert, nur das Band wird gestoppt $\to$ Wiederinbetriebnahme startet Band wieder - alle Funktionen wieder gegeben (CEO muss noch sagen wie das gemacht werden soll)
				\\ \hline
						
				
		\end{tabular} \\ \hline
	
	Ausnahmefälle &
		Später einfügen, Initial nur Normalverhalten \\ \hline
	Nachbedingungen & 
	 	\frqq UI.LED.stop\flqq leuchtet \\ \hline
	Zeitverhalten &
		...(muss hier etwas hin, wegen wenn zu früh oder zu spät?)  \\ \hline
	Verfügbarkeit & 
		...(So etwas wie erwartete / notwendige MTBF o.ä.) \\ \hline
	Fragen/Kommentare &
		Siehe \emph {Ausnahmefälle, Zeitverhalten, Verfügbarkeit} - das müssen wir noch besprechen \\ \hline
\end{tabular}
\newline
\newline
\end{table}


\begin{table}[htp]
\caption{WAVE LVL USE CASE}
\label{tab:usecase_wave}
\begin{tabular}{|p{0.3\linewidth}| p{0.7\linewidth} |}
	\hline 
	\rowcolor{tabgrey} \textbf{Name} & \textbf{Ein-Taste gedrückt wenn das Laufband aus ist} \\
	\hline

	Akteur & 
	\frqq UI.Taste.start\flqq \\ \hline
	Auslösendes Ereignis &
		\frqq UI.Taste.start\flqq werden durch mechanischen Eingriff ausgelöst \\ \hline
	Kurzbeschreibung & 
		Die Start-Taste wird gedrückt, Starten, Testlauf zur Kalibrierung startet... \\ \hline
		
	Vorbedingungen & 
		Wenn Zustand des Motors \flqq motor.stopp\frqq \\ \hline
	Essentielle Schritte &
		\begin{tabular}{|p{0.4\linewidth}|p{0.52\linewidth}|}
		\hline
			\rowcolor{tabgrey} \textbf{Intention \newline der Systemumgebung} & \textbf{Reaktion des Systems} \\ \hline \rowcolor{white}

			Schritt 1: \frqq UI.Taste.start\flqq löst ein Ereignis aus \newline führt zu \emph{Ein-Taste gedrückt wenn das Laufband aus ist} &
				Reaktion 1: \flqq Motor.rechtslauf\frqq wird für  \emph{Band1} und \emph{Band2} ausgelöst  \newline \\ \hline & 
				Reaktion 2: \frqq UI.LED.start\flqq Leuchtet \\ \hline 
				
		\end{tabular} \\ \hline
	
	Ausnahmefälle &
		Später einfügen, Initial nur Normalverhalten \\ \hline
	Nachbedingungen & 
	 	\frqq UI.LED.start\flqq leuchtet \\ \hline\\ \hline
	Zeitverhalten &
		...(muss hier etwas hin, wegen wenn zu früh oder zu spät?)  \\ \hline
	Verfügbarkeit & 
		...(So etwas wie erwartete / notwendige MTBF o.ä.) \\ \hline
	Fragen/Kommentare &
		Siehe \emph {Ausnahmefälle, Zeitverhalten, Verfügbarkeit} - das müssen wir noch besprechen \\ \hline
\end{tabular}
\newline
\newline
\end{table}


\begin{table}[htp]
\caption{WAVE LVL USE CASE}
\label{tab:usecase_wave}
\begin{tabular}{|p{0.3\linewidth}| p{0.7\linewidth} |}
	\hline 
	\rowcolor{tabgrey} \textbf{Name} & \textbf{Reset Taste wird gedrückt wenn das Laufband an ist} \\
	\hline

	Akteur & 
	\frqq UI.Taste.reset\flqq \\ \hline
	Auslösendes Ereignis &
		\frqq UI.Taste.reset\flqq wird durch mechanischen Eingriff ausgelöst \\ \hline
	Kurzbeschreibung & 
		Die Reset-Taste wird gedrückt \\ \hline
		
	Vorbedingungen & 
		Wenn Zustand des Motors \emph{nicht} \flqq motor.stopp\frqq :/ODERWIEoderwas?\\ \hline
	Essentielle Schritte &
		\begin{tabular}{|p{0.4\linewidth}|p{0.52\linewidth}|}
		\hline
			\rowcolor{tabgrey} \textbf{Intention \newline der Systemumgebung} & \textbf{Reaktion des Systems} \\ \hline \rowcolor{white}

			Schritt 1: \frqq UI.Taste.reset\flqq löst ein Ereignis aus \newline führt zu \emph{Reset Taste wird gedrückt wenn das Laufband an ist} &
				Reaktion 1: \frqq UI.LED.reset\flqq Leuchten	\newline \\ \hline 	
				
		\end{tabular} \\ \hline
	
	Ausnahmefälle &
		Später einfügen, Initial nur Normalverhalten \\ \hline
	Nachbedingungen & 
	 	...\\ \hline
	Zeitverhalten &
		...(muss hier etwas hin, wegen wenn zu früh oder zu spät?)  \\ \hline
	Verfügbarkeit & 
		...(So etwas wie erwartete / notwendige MTBF o.ä.) \\ \hline
	Fragen/Kommentare &
		Siehe \emph {Ausnahmefälle, Zeitverhalten, Verfügbarkeit} - das müssen wir noch besprechen \\ \hline
\end{tabular}
\newline
\newline
\end{table}

\begin{table}[htp]
\caption{WAVE LVL USE CASE}
\label{tab:usecase_wave}
\begin{tabular}{|p{0.3\linewidth}| p{0.7\linewidth} |}
	\hline 
	\rowcolor{tabgrey} \textbf{Name} & \textbf{Reset Taste wird gedrückt wenn das Laufband aus ist} \\
	\hline

	Akteur & 
	\frqq UI.Taste.reset\flqq \\ \hline
	Auslösendes Ereignis &
		\frqq UI.Taste.reset\flqq wird durch mechanischen Eingriff ausgelöst \\ \hline
	Kurzbeschreibung & 
		Die Reset-Taste wird gedrückt \\ \hline
		
	Vorbedingungen & 
		Wenn Zustand des Motors: \frqq motor.stopp \flqq :/mussdatso?\\ \hline
	Essentielle Schritte &
		\begin{tabular}{|p{0.4\linewidth}|p{0.52\linewidth}|}
		\hline
			\rowcolor{tabgrey} \textbf{Intention \newline der Systemumgebung} & \textbf{Reaktion des Systems} \\ \hline \rowcolor{white}

			Schritt 1: \frqq UI.Taste.reset\flqq löst ein Ereignis aus \newline führt zu \emph{Reset Taste wird gedrückt wenn das Laufband aus ist} &
				Reaktion 1: \frqq UI.LED.reset\flqq Leuchtet	\newline \\ \hline 	& Reaktion 2: Fehler quittiert $\to$ normaler Betrieb 
				
		\end{tabular} \\ \hline
	
	Ausnahmefälle &
		Später einfügen, Initial nur Normalverhalten \\ \hline
	Nachbedingungen & 
	 	...\\ \hline
	Zeitverhalten &
		...(muss hier etwas hin, wegen wenn zu früh oder zu spät?)  \\ \hline
	Verfügbarkeit & 
		...(So etwas wie erwartete / notwendige MTBF o.ä.) \\ \hline
	Fragen/Kommentare &
		Siehe \emph {Ausnahmefälle, Zeitverhalten, Verfügbarkeit} - das müssen wir noch besprechen \\ \hline
\end{tabular}
\newline
\newline
\end{table}

\begin{table}[htp]
\caption{WAVE LVL USE CASE}
\label{tab:usecase_wave}
\begin{tabular}{|p{0.3\linewidth}| p{0.7\linewidth} |}
	\hline 
	\rowcolor{tabgrey} \textbf{Name} & \textbf{E-Stopp Taste wird gedrückt wenn das Laufband an ist} \\
	\hline

	Akteur & 
	\frqq UI.Taste.eStop\flqq \\ \hline
	Auslösendes Ereignis &
		\frqq UI.Taste.eStop\flqq wird durch mechanischen Eingriff ausgelöst \\ \hline
	Kurzbeschreibung & 
		Die E-Stopp - Taste wird gedrückt, was tun?... \\ \hline
		
	Vorbedingungen & 
		... \\ \hline
	Essentielle Schritte &
		\begin{tabular}{|p{0.4\linewidth}|p{0.52\linewidth}|}
		\hline
			\rowcolor{tabgrey} \textbf{Intention \newline der Systemumgebung} & \textbf{Reaktion des Systems} \\ \hline \rowcolor{white}

			Schritt 1: \frqq UI.Taste.eStopp\flqq löst ein Ereignis aus \newline führt zu \emph{E-Stopp Taste wird gedrückt wenn das Laufband an ist} &
				Reaktion 1: Laufband-Motor stoppen, Lampen aus?	\newline \\ \hline 	
				
		\end{tabular} \\ \hline
	
	Ausnahmefälle &
		Später einfügen, Initial nur Normalverhalten \\ \hline
	Nachbedingungen & 
	 	...\\ \hline
	Zeitverhalten &
		...(muss hier etwas hin, wegen wenn zu früh oder zu spät?)  \\ \hline
	Verfügbarkeit & 
		...(So etwas wie erwartete / notwendige MTBF o.ä.) \\ \hline
	Fragen/Kommentare &
		Siehe \emph {Ausnahmefälle, Zeitverhalten, Verfügbarkeit} - das müssen wir noch besprechen \\ \hline
\end{tabular}
\newline
\newline
\end{table}


\begin{table}[h]
\caption{WAVE LVL USE CASE}
\label{tab:usecase_wave}
\begin{tabular}{|p{0.3\linewidth}| p{0.7\linewidth} |}
	\hline 
	\rowcolor{tabgrey} \textbf{Name} & \textbf{E-Stopp Taste wird gedrückt wenn das Laufband aus ist} \\
	\hline

	Akteur & 
	\frqq UI.Taste.eStop\flqq \\ \hline
	Auslösendes Ereignis &
		\frqq UI.Taste.eStop\flqq wird durch mechanischen Eingriff ausgelöst \\ \hline
	Kurzbeschreibung & 
		Die E-Stopp - Taste wird gedrückt, was tun?... \\ \hline
		
	Vorbedingungen & 
		Wenn Zustand des Motors: \frqq motor.stopp \flqq \\ \hline
	Essentielle Schritte &
		\begin{tabular}{|p{0.4\linewidth}|p{0.52\linewidth}|}
		\hline
			\rowcolor{tabgrey} \textbf{Intention \newline der Systemumgebung} & \textbf{Reaktion des Systems} \\ \hline \rowcolor{white}

			Schritt 1: \frqq UI.Taste.eStopp\flqq löst ein Ereignis aus \newline führt zu \emph{E-Stopp Taste wird gedrückt wenn das Laufband aus ist} &
				Reaktion 1: Lampen aus, was noch? Wiederinbetriebnahme mit Reset-quittierung?	\newline \\ \hline 	
				
		\end{tabular} \\ \hline
	
	Ausnahmefälle &
		Später einfügen, Initial nur Normalverhalten \\ \hline
	Nachbedingungen & 
	 	...\\ \hline
	Zeitverhalten &
		...(muss hier etwas hin, wegen wenn zu früh oder zu spät?)  \\ \hline
	Verfügbarkeit & 
		...(So etwas wie erwartete / notwendige MTBF o.ä.) \\ \hline
	Fragen/Kommentare &
		Siehe \emph {Ausnahmefälle, Zeitverhalten, Verfügbarkeit} - das müssen wir noch besprechen \\ \hline
\end{tabular}
\newline
\newline
\end{table}


\begin{table}[htp]
\caption{WAVE LVL USE CASE}
\label{tab:usecase_wave}
\begin{tabular}{|p{0.3\linewidth}| p{0.7\linewidth} |}
	\hline 
	\rowcolor{tabgrey} \textbf{Name} & \textbf{Reset Taste wird gedrückt wenn das Laufband durch E-Stopp gestoppt wurde} \\
	\hline

	Akteur & 
	\frqq UI.Taste.reset\flqq \\ \hline
	Auslösendes Ereignis &
		\frqq UI.Taste.reset\flqq wird durch mechanischen Eingriff ausgelöst \\ \hline
	Kurzbeschreibung & 
		Die Reset-Taste wird gedrückt \\ \hline
		
	Vorbedingungen & 
		Wenn Zustand des Motors: \frqq motor.stopp \flqq, oder wat??? \\ \hline
	Essentielle Schritte &
		\begin{tabular}{|p{0.4\linewidth}|p{0.52\linewidth}|}
		\hline
			\rowcolor{tabgrey} \textbf{Intention \newline der Systemumgebung} & \textbf{Reaktion des Systems} \\ \hline \rowcolor{white}

			Schritt 1: \frqq UI.Taste.reset\flqq löst ein Ereignis aus \newline führt zu \emph{Reset Taste wird gedrückt wenn das Laufband durch E-Stopp gestoppt wurde} &
				Reaktion 1:  \frqq motor.rechtslauf\flqq 	\newline \\ \hline 	& Reaktion 2:  \frqq Ampel.gruen\flqq \newline \\ \hline 	
				& Reaktion 3:  \frqq UI.LED.start\flqq \newline \\ \hline 	
				& Reaktion 4:  Startroutine beginnt halt
				
		\end{tabular} \\ \hline
	
	Ausnahmefälle &
		Später einfügen, Initial nur Normalverhalten \\ \hline
	Nachbedingungen & 
	 	...\\ \hline
	Zeitverhalten &
		...(muss hier etwas hin, wegen wenn zu früh oder zu spät?)  \\ \hline
	Verfügbarkeit & 
		...(So etwas wie erwartete / notwendige MTBF o.ä.) \\ \hline
	Fragen/Kommentare &
		Siehe \emph {Ausnahmefälle, Zeitverhalten, Verfügbarkeit} - das müssen wir noch besprechen \newline Nils ist schuld \\ \hline
\end{tabular}
\newline
\newline
\end{table}

\end{document}
% vim: set spell spelllang=de :EOF
