\documentclass[10pt]{article}
%Gummi|065|=)

\usepackage[ngerman,english]{babel}
%% see http://www.tex.ac.uk/cgi-bin/texfaq2html?label=uselmfonts
\usepackage[T1]{fontenc}
\usepackage[utf8]{inputenc}
%\usepackage[latin1]{inputenc}
\usepackage{libertine}
\usepackage{pifont}
\usepackage{microtype}
\usepackage{textcomp}
\usepackage[german,refpage]{nomencl}
\usepackage{setspace}
\usepackage{makeidx}
\usepackage{listings}


\usepackage{longtable}
\usepackage{enumitem}
\usepackage[hyphens]{url}
\urlstyle{same}
\title{\textbf{Begriffsglossar ESE-Projektaufgabe}}
\author{Nils Eggebrecht\\
		Lennart Hartmann\\
		Alexander Mendel\\
		Eduard Veit\\
		Karl-Fabian Witte}
\date{[30. März 2017]}

\begin{document}

\newenvironment{sourceenum}{
  \vspace{-\baselineskip}
    \begin{enumerate}[leftmargin=*,noitemsep,topsep=0pt,partopsep=0pt]}
   {\end{enumerate}}
\maketitle
\texttt {Dieses Dokument ist vorerst Abgeschlossen. Es werden keine Änderungen benötigt.}
\section{Glossar}

\begin{longtable}{|p{0.2\textwidth} p{0.9\textwidth} |}
	\hline
	\textbf{Begriff} & \textbf{Erläuterung} \\ [5pt]
	\hline
	\endhead
	\hline 
	\endfoot
	

\\ \hline
	EntryManager &
	Dies ist der Zustandsautomat, der die erste Lichtschranke behandelt. Zustände sind \newline\textbf{Ready}: \newline  Es findet eine Transition statt, wenn die Methode ReportNewPuk im Inputmultiplexer aufgerufen wird. Daraufhin wird ein Puk in der Liste angelegt
\\ \hline

\end{longtable}
\end{document}
\grid
