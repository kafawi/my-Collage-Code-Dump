% PROTOKOLL Action Item
\documentclass[
   draft=false
  ,paper=a4
  ,twoside=false
  ,fontsize=11pt
  ,headsepline
  ,DIV11
  ,parskip=full+
]{scrartcl} % copied from Thesis Template from HAW

\usepackage[ngerman,english]{babel}
\usepackage[T1]{fontenc}
\usepackage[utf8]{inputenc}

\usepackage[
    left  =4em
   ,right =4em
   ,top   =5em
   ,bottom=5em
]{geometry}

\usepackage{longtable}
\usepackage[german,refpage]{nomencl}

\usepackage{float}
\usepackage{enumitem}
\usepackage{hyperref} % for a better experience

\hypersetup{
   colorlinks=true % if false - links get colored frames
  ,linkcolor=black % color of tex intern links
  ,urlcolor=blue   % color of url links
}

\usepackage{amsmath}

\usepackage{array}   % for \newcolumntype macro
\newcolumntype{L}{>{$}l<{$}} % math-mode version of "l" column type
\newcolumntype{R}{>{$}r<{$}} % math-mode version of "r" column type
\newcolumntype{C}{>{$}c<{$}} % math-mode version of "c" column type

\usepackage{listing}
\usepackage{caption}
\usepackage{colortbl}
\definecolor{tabgrey}{rgb}{0.85,0.85,0.85}
%using minted because of the hashtag in bash

\sloppy
\clubpenalty=10000
\widowpenalty=10000
\displaywidowpenalty=10000

\begin{document}

\selectlanguage{ngerman}
% ----------------------------------------------------------------------------
% ---------------------------------------------------------- HIER WAS MACHEN -
% -------------------------------- Metadaten wie namen und Gruppentreffen etc-
\def\titel{Requirements - Puks am 04.04.2017}

\def\myName{Karl-Fabian Witte}
\def\myEmail{Karl-Fabian.Witte@haw-hamburg.de}


\def\teilnehmer{ 
	& Lennart Hartmann & Eduard Veit \\
	& Alexander Mendel   & Nils Eggebrecht\\
    & Karl-Fabian Witte   & \\
}

\def\zurKenntnis {
	& T.Lehmann(HAW Hamburg) & Profs(HAW Hamburg) \\
}
% -------------------------------------------------- HIER AUFHÖREN ----------



% ------------------------------------------ einige strukturell Definitionen
\newlength{\txtw} %definiere neue länge
\setlength{\txtw}{\textwidth} %setze neue länge auf textbreite
\addtolength{\txtw}{-10\tabcolsep} %subtrahiere -8\cdot textbreite von asdf

\def\me{\myName \newline \footnotesize{\url{\myEmail} } }

\def\tablehead{
	\hline 
	\rowcolor{tabgrey}
	\textbf{Typ} & 
	\textbf{Resultat regulär} & 
	\textbf{gewünschte Rutsche voll} \\
	\hline 
	\endhead}
	
% ------------------------------------------------------------------ Inhalt	
\begin{tabular}{p{0.65\txtw} p{0.35\txtw}}
	\textbf{\Large{\titel}} & \me  \\
\end{tabular}

\begin{tabular}{l p{0.4\txtw} p{0.4\txtw} }
	Teilnehmer: \teilnehmer
	Zur Kenntnis: \zurKenntnis
\end{tabular}
\texttt {Dieses Dokument ist vorerst Abgeschlossen. Es werden keine Änderungen benötigt.}
\subsection*{\titel}
\vspace{-1em}

\begin{longtable}{
	|p{0.30\txtw} % Nr.
	|p{0.30\txtw} % Art
	|p{0.30\txtw} % Sichwort und Beschreibung
|}
	\tablehead

% ----------------------------------------------------------------------------
% ---------------------------------------------------------- HIER WAS MACHEN -
% --------------------------------------- hier kommen die Protokollpunkte hin
% ---------------- mit \newline wird in der Tabelle ein Zeilenumbruch gemacht
 
	Bohrung unten & Band1/Band2 & Band2 \\ \hline

	Flach & Band1 (und gelb blinkt) & Band2 (und gelb blinkt) \\ \hline


	Bohrung & falls Sequenz falsch: Band2 & - \\ \hline

	Bohrung-Metall & falls Sequenz falsch: Band2 & -  \\ \hline		

	Typ 1/5 & Band1 & Band2 \\ \hline

	Typ 2/4 & Band2 & Band1 \\ \hline

	unbek. Lieg Objekt-ULO & Band1/Band2 & Band2/Band1 \\ \hline

% -------------------------------------------------- HIER AUFHÖREN ----------
% ---------------------------------------------------------------------------

\end{longtable}

\vspace{-2.5em}
\footnotesize

\normalsize

% ----------------------------------------------------------------------------
% ---------------------------------------------------------- HIER WAS MACHEN -
%---------------------------------------------- Infos für den nächsten Termin

% -------------------------------------------------- HIER AUFHÖREN ----------

% ----------------------------------------------------------------------------
% ----------------------------------------------------------------------------
\flushleft

\end{document}
% vim: set spell spelllang=de :EOF