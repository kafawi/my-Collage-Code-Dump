% PROTOKOLL Action Item
\documentclass[
   draft=false
  ,paper=a4
  ,twoside=false
  ,fontsize=11pt
  ,headsepline
  ,DIV=11
  ,parskip=full+
  ,titlepage
]{scrartcl} % copied from Thesis Template from HAW

\usepackage[ngerman,english]{babel}
\usepackage[T1]{fontenc}
\usepackage[utf8]{inputenc}

\usepackage[
    left  =4em
   ,right =4em
   ,top   =5em
   ,bottom=5em
]{geometry}

\usepackage{longtable}
\usepackage[german,refpage]{nomencl}

\usepackage{float}
%\usepackage{enumitem}
\usepackage{paralist} %compactitem
\usepackage{hyperref} % for a better experience

\usepackage{scrpage2}
\pagestyle{scrheadings}
\clearscrheadfoot
\ohead{\pagemark}
\ihead{\headmark}
\automark{section}

\hypersetup{
   colorlinks=true % if false - links get colored frames
  ,linkcolor=black % color of tex intern links
  ,urlcolor=blue   % color of url links
}

\usepackage{amsmath}

\usepackage{array}   % for \newcolumntype macro
\newcolumntype{L}{>{$}l<{$}} % math-mode version of "l" column type
\newcolumntype{R}{>{$}r<{$}} % math-mode version of "r" column type
\newcolumntype{C}{>{$}c<{$}} % math-mode version of "c" column type

\usepackage{color}
\definecolor{black}{rgb}{0,0,0}
\definecolor{darkgray}{rgb}{0.2,0.2,0.2}
\definecolor{lightgray}{rgb}{0.9,0.9,0.9}
\definecolor{blue}{rgb}{0.0,0.0,0.9}
\definecolor{orange}{rgb}{0.8,0.4,0.0}
\definecolor{green}{rgb}{0.0,0.7,0.0}
\definecolor{red}{rgb}{0.9,0.0,0.0}

\usepackage{listings, lstautogobble}
\lstset{%
   language=[11]C++
  ,frame=lines
  ,numbers=left
  ,numberstyle=\tiny\color{darkgray}
  ,stepnumber=1
  ,numbersep=5pt
  ,backgroundcolor=\color{lightgray}
  ,showspaces=false
  ,keepspaces=true
  ,autogobble=true 
  ,breaklines=true
  ,tabsize=2
  , 
  ,basicstyle=\footnotesize\ttfamily\color{black}
  ,identifierstyle=\color{orange}
  ,keywordstyle=[1]\color{blue}\textbf
  ,keywordstyle=[2]\color{red}\textbf
  ,stringstyle=\color{green}
  ,commentstyle=\color{darkgray}\textit
}
  
\usepackage{caption}
\usepackage{colortbl}
\definecolor{tabgrey}{rgb}{0.85,0.85,0.85}
%using minted because of the hashtag in bash

\sloppy
\clubpenalty=10000
\widowpenalty=10000
\displaywidowpenalty=10000


% set font 
\renewcommand{\familydefault}{\sfdefault}
\usepackage{times}

\begin{document}

\selectlanguage{ngerman}
% ----------------------------------------------------------------------------
% ---------------------------------------------------------- HIER WAS MACHEN -
% -------------------------------- Metadaten wie namen und Gruppentreffen etc-
\title{C++ Coding Style}
\subtitle{LANKE}
\date{\today}

\publishers{%
	\normalfont\normalsize%
	\parbox{0.8\linewidth}{\centering
		Um die Garantie zu geben, dass die Qualität des Quellcodes
		eigehalten wird, ist eine Richtlinie für den Code zu schreiben,
		unabdingbar. Auf Basis von den Codebeispielen der Professoren,
		unseren Erfahrungen und allgemeinen Vereinbarungen in der 
		Programmierer Szene ist hier für das Praktikum ESE eine solche 
		Richtline erstellt worden.
	}
}

\maketitle
\setcounter{page}{1}
\tableofcontents
\flushleft

\section{Einleitung}
	Dieses Dokument legt die Konventionen für den Quellcode fest.
	Zuerst wird das Sprachbild ( die Namensgebung, das Layout
	und die Struktur im Code ) festgelegt. Danach wird über das
	Dokumentieren gesprochen. Das Testen eines Modules wird abschließend
	besprochen.
	
\section{C++ Sprachbild}
\subsection{Namengebung}

  \begin{minipage}[t]{0.45\linewidth}
  Alle Namen sind Englisch. 
  \begin{compactitem}
  \item[\textbf{Typen:}] \textbf{C}amel\textbf{C}ase 
    \begin{compactitem}
    \item Klasse (class)
    \item Struktur (struct)
    \item Aufzählung (enum)
      \begin{compactitem}
      \item Singular (Status statt Stati)
      \end{compactitem}
    \item typedef
    \end{compactitem}
  \item[\textbf{Variablen:}] \textbf{c}amel\textbf{C}ase
  \item[\textbf{private Attribute:}] \textbf{c}amel\textbf{C}ase
  \begin{compactitem}
    \item suffix\textbf{\_} 
  \end{compactitem}  
  \item[\textbf{Konstanten:}] \textbf{UPPER\_CASE} 
  \item[\textbf{Funktionen:}] Verb + Nomen, \textbf{c}amel\textbf{C}ase 
  \item[\textbf{Flags:}] \textbf{c}amel\textbf{C}ase
    \begin{compactitem}
    \item \textbf{is}Prefix, 
    \item keine Negation 
    \item > isRunning anstatt isNotRunning
    \end{compactitem} 
  \end{compactitem}
  \end{minipage}%
  \hfill
  \begin{minipage}[t]{0.5\linewidth}
    \begin{lstlisting}
      #define UNIVERCITY_NAME "HAW Hamburg" 
      class UniAccount
      {
        public:
          Status getStatus(void) ;
      
        private:
          Status status_ ;
          bool isStudent_ ;
      }
      
      enum Status 
      {
         STUDENT
        ,PROF
      }     
    \end{lstlisting}
  \end{minipage}%

\subsection{Layout}
\begin{minipage}[t]{0.45\linewidth}

  \large{\textbf{white space}}\normalsize 
  \begin{compactitem}
    \item[\textbf{Einrückung:}] für jeden Block
    \begin{compactitem}
      \item Leerzeichen statt Tabs
      \item 2 oder 3 Leerzeichen
    \end{compactitem}
    \item[\textbf{Zeilen zwischen...}]\textbf{:}
    \begin{compactitem}
      \item[Funktionen:] eine Leerzeile
      \item[logische Aufteilung:] Zeilenkommentar statt Leerzeile
    \end{compactitem}
    \item[\textbf{Lesbarkeit:}] ist das oberste Ziel
    \begin{compactitem}
      \item \textbf{Klammern:} je nach Lesbarkeit 
      \item \textbf{Operantionen:} zwischen Operanten 
      und Operatoren ein Leerzeichen
      \item \textbf{Listen:} hinter jedem Komma ein Leerzeichen 
      (Ausnahme: wenn Liste unteinander geschrieben wird )
      \item \textbf{Zuweisungen:} es dürfen mit Leerzeichen Tabellen gemimt werden.
    \end{compactitem}
  \end{compactitem}
  
  \large{\textbf{Blöcke}}\normalsize 
  \begin{compactitem}
  \item[\textbf{öffnende Blockklammer:}] $\{$
  
  \begin{compactitem}
    \item \textbf{Typen:} in neuer Zeile
    \item \textbf{Schleifen:} direkt dahinter oder neue Zeile
    \item \textbf{Bedingungen:} direkt dahinter oder neue Zeile
  \end{compactitem}
  \item[\textbf{schließende Blockklammer:}] $\}$
    \item in der Einrückebene, in der sie geöffnet wurden
    \item sind einziges Zeichen in Zeile
  \end{compactitem}
  
  \large{\textbf{Zeilenumbruch}}\normalsize 
  \begin{compactitem}
    \item[\textbf{Zeilenzeichenlimit:}] 80 Zeichen
    \item[\textbf{Deklaration:}] je eine Zeile
    \item wenn das Zeichenlimit pro Zeile nicht reicht...
    \begin{compactitem}
      \item \textbf{Parameter-/Argumentenliste:} wo die Klammer sich öffnet
      \item \textbf{Zuweisung, Berechnung:} wo $=$ beginnt
    \end{compactitem}
  \end{compactitem}
  \end{minipage}%
  \hfill
  \begin{minipage}[t]{0.5\linewidth}
    \begin{lstlisting}
      class UniAccount
      {
        public:
          void work(int time){
            // get infos 
            int freeTime      = freeTime_ ;
            Status motivation = getMotivation();
            // test if she/he is capable to work
            if (freeTime > time && motivation => OK){
              workTime_ += time;
              freeTime_ -= time;
            }
            return;
          }
        
          int calcSomthing ( int a, int b, int c, 
                             int d, int e, inf f 
          ){
            int thisReturnValueNameLong = a + b + c 
                                        + d + e + f;
            return thisReturnValueNameLong;    
          }
        
        private:
          workTime_ ;
          freeTime_ ;
      }
      
      enum Status
      {
         DEPRESSED
        ,OK
        ,MOTIVATED
      } 
      
    \end{lstlisting}    
  \end{minipage}%

\section{Struktur}

\begin{minipage}[t]{0.45\linewidth}
  \begin{compactitem}
    \item[\textbf{Module:}] Teile Implementation vom Interface
    \begin{compactitem}
      \item \textbf{Module.h:} Interface
      \begin{compactitem}
        \item verwende include guards (MODULE\_H\_)
        \item Deklarationen
        \item Templates
        \item inline function Definitionen
      \end{compactitem}
      \item \textbf{Module.cpp:} Implementation
       \begin{compactitem}
        \item Funktionsdefinitionen
        \item strikte innere Klassen
      \end{compactitem}
      \item \textbf{Ausnahme:} Eigene Bibliotheken sind in einem 
        Header definiert.
    \end{compactitem}
    \item[\textbf{Weiteres:}] Was soll noch eingehalten werden.
    \begin{compactitem}
        \item keine magic numbers (lieber Konstanten)
        \item vermeide namespaces
        \item order of includes: most spacific first
        \item wenn du einen Header nur im cpp file benötigst,
        füge diesen auch nur dort ein
        \item wenn möglich, verwende forward declaration
        \item erstelle den Ctor
        \item verwende RAII (Ressource hängt von Lebzeit des Objekts ab)
        \item versuche die rule of three
        \begin{compactitem}
          \item Konstruktor und passender Destruktor  
          \item Kopierkonstruktor
          \item assigment operator
        \end{compactitem}
      \end{compactitem}
  \end{compactitem}
  \end{minipage}%
  \hfill
  \begin{minipage}[t]{0.5\linewidth}
    Module.h
    \begin{lstlisting}
      #ifndef MODULE_H_
      #define MODULE_H_
      
      class SomeClass;

      class Module
      {
        public:
          // rule of three
          Module(SomeClass) ; // ctor
          virtual ~Module() ; // destructor
          Module(const Module&) ; // copy constructor
          Module& operator= (const Module&); // assigment operator
        
          void printSomething(void) const;
        
        private:
          SomeClass &handle_ ;
      }
      #endif /* MODULE_H_ */
      
    \end{lstlisting}    
    Module.cpp
    \begin{lstlisting}
      #include "Module.h"
      #include "SomeClass.h"
      
      #include <iostream>
      
      using namespace std;
      
      // RAII begin
      Module::Module(SomeClass cl)
      : handle_(cl)
      {
        //ctor
      }
      
      Module::~Module(){
        release (handle_);
      }
      // RAII end
      
      Module::printSomething(void){
        cout << handle_->getSomthing() << endl;
      }
      
    \end{lstlisting}  
  \end{minipage}%

\newpage
\section{Dokumentation}
	
	\begin{minipage}[t]{0.45\linewidth}
	  \large{\textbf{CODE DOKUMENTATION}}\newline
	  \normalsize
	  Die Dokumentation wird mit dOxygen erzeugt. Dafür wird mit Tags 
		auf Informationen gesammelt.
		DOxygen holt sich die Informationen aus 
		nebenstehenden Kommentarstrukturen. Funktionen und Typen werden mit dem
		Block erklärt. Variablen, Attribute und Konstanten werden mit
		$/**< ... */$ beschrieben. Die Beschreibung erfolgt fast ausschließlich 
		in der Headerdatei des Modules.
		Erstellt wird die Dokumentation mit den Befehlen:
		\begin{lstlisting}
    $ doxygen <config-file>
		\end{lstlisting}
		
  \end{minipage}%
  \hfill
  \begin{minipage}[t]{0.5\linewidth}
    \begin{lstlisting}
    
    /** @file <filename>
     *  @brief <short description of module>
     *
     *  <detailes, longer explaination, pattern, componente>
     *  @author <author 1>
     *  @author <author 2>
     */
     
    /**
     * @class <description of class>
     */
    class Module
     { 
     ...
		/** 
		 *  @details description of function
		 *  @param a <description of 1st parameter>
		 *  @param b 
		 *      <description of 2nd parameter>
		 *  @return <description of retrunvalue>
		 */
		int  func(int a, int b);

		/// single line 
		
		int x_; /**< decription of x */
		\end{lstlisting}
  \end{minipage}%
	
		
		Die verwendetet Tags für dieses Projekt sind:
		\begin{table}[H]
			\begin{tabular}{|p{0.15\textwidth}|p{0.5\textwidth}|}
			\hline
			\rowcolor{lightgray}
			\textbf{tag}  & \textbf{Rendering} \\ \hline
			
			@file        & Name des Files                                  \\ \hline
			@brief       & kurze Beschreibung des Moduls                   \\ \hline
			@author      & Name des Authors                                \\ \hline
			@class \newline
			@enum \newline
			@struct       & Beschreibung des Typen                         \\ \hline
			@details     & kurze Beschreibung der Funktion                 \\ \hline
			@param <par> & Beschreibung des Parameters <par> in Funktionen \\ \hline
			@return      & Beschreibung des Rückgabewertes                 \\ \hline
			
			\end{tabular}
		\end{table} 
		
		
	\begin{minipage}[t]{0.45\linewidth}
	  \large\textbf{LIZENZ}\newline \normalsize
		Unser Project läuft unter der MIT Lizenz, welche in der Textdatei 
		\texttt{LICENSE.txt} beschrieben ist. Jedes File muss den folgenden
		Text im header haben.
		
  \end{minipage}%
  \hfill
  \begin{minipage}[t]{0.5\linewidth}
    \begin{lstlisting}
        /**
         *   ...
         *  Embedded System Engineering SoSe 2017
         *  Copyright (c) 2017  LANKE devs
         *  This software is licensed by MIT License.
         *  See LICENSE.txt for details.
         */
		\end{lstlisting}
  \end{minipage}%
  
\section{Testen}

Wie ein Unittest aussieht, ist jedem Entwickler freigestellt. Er hat 
somit die alleinige Verantwortung, dass sein Code richtig funktioniert. 
Die Testfiles müssen in dem Unterverzeichnis \texttt{test} abgelegt werden,
und sollten im Team veröffentlicht werden. 
Die Tests sollten alle Ausnahmefälle und die Funktion des Modules Testen.
Eventuell werden auch mehrere Komponenten gleichzeitig getestet. 
Bevor es an die Hardware geht, sollte der Code in Software reibungslos
laufen.

Mögliche Testwerkzeuge sind einmal ein Testmodul mit Mainfunktion oder 
man verwedet C++Unit.  

\section*{kurze Empfehlungen}
Wenn Pattern verwendet werden, sollten die Namen der Pattern bzw. 
deren Komponenten in den Namen der Klassen wiederzufinden sein. 


\end{document}